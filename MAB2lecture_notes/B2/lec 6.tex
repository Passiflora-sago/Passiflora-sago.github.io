\setcounter{chapter}{5}
\chapter{多元函数的极限与连续性}

\section{多元函数的例子}

多元函数形如$u = f(x_1,x_2,\cdots,x_n)$,其中$x_1,x_2,\cdots,x_n$是自变量,$u$是因变量.

\begin{enumerate}
    \item $z = ax+by+c, (x,y) \in \R^2 = \{ (x,y) | x,y \in \R, x^2 + y^2 \les +\infty \}$: 平面方程;
    \item $z = \sqrt{R^2 - x^2-y^2}, (x,y) \in D: x^2 + y^2 \les R^2$: 上半球面;
    \item $f(x,y) = \left( \frac{1}{\sqrt{2\pi}} \right)^2 \e^{-\frac{x^2+y^2}{2}}$: 二元正态分布概率密度函数;
    \item $u = \ln(a^2 - x^2-y^2-z^2), x^2+y^2+z^2 < a^2, \Omega: x^2+y^2+z^2 < a^2$为开球体;
    \item $B(x,y) = \int_0^1 t^{x-1} (1-t)^{y-1} \dif t, x,y > 0$: 贝塔函数.
    \item $u = a_1 x_1 + a_2 x_2 + \cdots + a_n x_n$: $n$元线性函数.
    \item $Q(x_1,x_2,\cdots,x_n) = \sum_{i=1}^n \sum_{j=1}^n a_{ij} x_i x_j, a_{ij} = a_{ji}$: $x_1,x_2,\cdots,x_n$的二次齐次函数.
\end{enumerate}

多元函数中,最简单的是二元函数$z=f(x,y), (x,y) \in D.$ 且$z = f(x,y)$有直观图像 --- 空间的曲面.因此,二元函数是今后的重点讨论的多元函数.

\section{平面点集的若干概念}

二元函数$z = f(x,y)$的定义域$D$是平面$\R^2$的一个子集.

\begin{enumerate}
    \item 点$M_0$的$\delta$邻域$\overline U(M_0,\delta) := \{ M : |MM_0| = \rho(M,M_0) < delta \}$,
    即$\overline U(M_0,\delta) = \{ (x,y) | (x-x_0)^2 + (y-y_0)^2 < \delta^2 \} \subset D$.
    \item $D$的内点$M_0$ : $M_0 \in D$,且$\exists \delta > 0 $,使得$\overline U(M_0,\delta) \subset D$.
    \item $D$的外点$M_0$ : $M_0 \notin D$,且$\exists \delta > 0 $,使得$\overline U(M_0,\delta) \cap D = \varnothing$.
    \item $D$的边界点$M_0$ : $M_0$的任意$\delta$邻域中都同时含有$D$中点与$D^c$中点.点集$D$的边界点全体记作$\partial D: D$的边界.
    \item 由全体内点组成的点集称为开集,开集$D$的余集$D^c$称为闭集.闭集的余集是开集.
    \item 连通性:若$D$中任意两点$A,B$都可以用$D$中连续曲线连接,则称$D$是联通的.
    \item 开集若是联通的,称之为开区域,简称为区域,开区域$D$与$D$的边界$\partial D$之并,称之为闭区域,记作$\overline D = D \cup \partial D$.
    \begin{remark}
    讲义上此处写为$\overline{D} = D + \partial D$.两种写法是等价的.
\end{remark}
    \item 若$\exists R >0$,使得$D \subset \overline{U}(0,R)$,则称$D$是有界集.
\end{enumerate}

\begin{example}
    $\overline U(M_0,\delta), \R^2, x^2+y^2+z^2<a^2$都是开集,$\overline U(M_0,\delta)^c, x^2+y^2+z^2 < a^2$是有界集,$\R^2$是无界集.
    $(x-x_0)^2 + (y-y_0)^2 \ges \delta^2, (R^2)^c = \varnothing, x^2+y^2+z^2 \ges a^2$是闭集.
    $(x-x_0)^2 + (y-y_0)^2 \les \delta^2, x^2+y^2+z^2 \les a^2$是有界闭集.
\end{example}

\begin{example}
    空集$\varnothing$由零个内点组合,因此是开集;$\varnothing^c = \R^2$开,因此$\varnothing$是闭集.
    在所有点集之中,只有空集和全集是既开又闭的.
\end{example}

\section{二元函数$f(x,y)$的极限与连续性}

\begin{enumerate}
    \item 若$\forall \delta >0, \overline U(M_0,\delta)$都有点集$D$中点,则称$M_0$是$D$原点(极限点),$M_0$这个原点可以属于$D$,也可以不属于$D$.
    \item 设点$M_0 \in D$,且$\exists \delta > 0 $,使得$\overline U(M_0,\delta)$中除$M_0$无$D$中点,则称$M_0$是$D$的孤立点.
\end{enumerate}

\begin{definition}
    设 \( z = f(x, y) \) 是定义在平面点集 \( D \) 上的二维函数,\( M_0 = (x_0, y_0) \) 是 \( D \) 的聚点,又设 \( a \) 是一个数.如果对任意给定的 \( \epsilon > 0 \),存在 \( \delta > 0 \),当 \( M = (x, y) \in D \) 满足

$$
0 < \rho(M, M_0) = \sqrt{(x - x_0)^2 + (y - y_0)^2} < \delta
$$

或者
$$
0 < |x - x_0| < \delta, \quad 0 < |y - y_0| < \delta
$$

时,有
$$
|f(M) - a| < \varepsilon,
$$

那么称当 \( M \) 趋于 \( M_0 \) 时 \( f(M) \) 以 \( a \) 为极限,记作
$$
\lim_{M \to M_0} f(M) = a.
$$

也可以写成
$$
\lim_{(x, y) \to (x_0, y_0)} f(x, y) = a \quad \text{或} \quad \lim_{x \to x_0 , y \to y_0} f(x, y) = a.
$$
\end{definition}

由于多元函数的极限与一元函数的极限定义的方式相同.因此,一元函数极限中的四则运算法则,夹逼准则,及极限的唯一性,局部有界性,保号性,保序性等都可以推广到多元函数的极限之中来.

\begin{definition}
    设 \( f(x, y) \) 在 \( (x_0, y_0) \) 的邻域 \( B(M_0, r) = \{ M \mid \rho(M, M_0) < r \} \) 有定义, 如
$$
\rho = \sqrt{(x - x_0)^2 + (y - y_0)^2} < \delta,
$$

或者
$$
|x - x_0| < \delta, \quad |y - y_0| < \delta
$$

时, 就有
$$
|f(x, y) - f(x_0, y_0)| < \varepsilon
$$

也就是说极限
$$
\lim_{x \to x_0, y \to y_0} f(x, y) = f(x_0, y_0),
$$

或
$$
\lim_{\Delta x \to 0, \Delta y \to 0} f(x_0 + \Delta x, y_0 + \Delta y) = f(x_0, y_0),
$$

那么称 \( f \) 在 \( (x_0, y_0) \) 连续.如果 \( f \) 在区域 \( D \) 的每一个点连续,就称 \( f \) 在 \( D \) 上连续.
\end{definition}

\begin{remark}
    多元函数的一致连续性指的是$\delta$与$\varepsilon$与点$M_0$无关,具体而言,若$\forall \varepsilon > 0, \exists \delta > 0$,使得$\forall M_1,M_2 \in D$,当$\rho(M_1,M_2) < \delta$时,有$|f(M_1)-f(M_2)| < \varepsilon$,则称$f(x,y)$在$D$上一致连续.
\end{remark}

从定义可知,若$M_0$是$D$是原点,则必有$\lim_{M \to M_0} f(M) = f(M_0) = f(\lim_{M \to M_0} M)$,即极限号与函数符号可交换.

若$M_0(x_0,y_0)$是$D$的孤立点,则$f(M)$在$M_0$处必连续.

\begin{proof}
    $\forall \ve >0, \exists \delta >0$,使得$\overline U(M_0,\delta)$中除$M_0$外无$D$中点.当$M \in D, | MM_0 | < \delta $时,$|f(M) - f(M_0)| = 0 < \ve$,即$\lim_{M \to M_0} f(M) = f(M_0)$.
\end{proof}

\begin{remark}      
    此处使用的是课本上的定义方式.
\end{remark}

\begin{example}
    $f(x,y) = \sqrt{\cos^2 \pi x + \cos^2 \pi y -2}$的定义域$D$由所有的整点(格点)$M(m,n),m,n \in \Z$组成.每个整点都是$D$的孤立点.也都是$f(x,y)$的连续点,从而$f(x,y)$在$D$上连续.
\end{example}

\begin{example}
    考察下列极限:
    \begin{enumerate}
        \item 证明:$\lim_{x \to 0, y \to 0} \frac{x^2 y^2}{x^4 + y^2} =0$;
        \item 证明:$\lim_{x \to 0, y \to 0} \frac{x^2 y}{x^4 + y^2} $不存在;
        \item 证明:$\lim_{x \to 0, y \to 0} \left( 1 +xy \right)^{\frac{1}{x+y}} $不存在.
    \end{enumerate}
\end{example}

\begin{proof}
    \begin{enumerate}
        \item $0 \les \frac{x^2 y^2}{x^4 + y^2} \les \frac12 |y|$,且$\lim_{x \to 0, y \to 0} 0 = 0 = \lim_{x \to 0, y \to 0} \frac12 |y|$,由夹逼准则,得$\lim_{x \to 0, y \to 0} \frac{x^2 y^2}{x^4 + y^2} =0$;
        \item 取$y = kx^2$,$k$为常数,即动点$M(x,y)$沿抛物线$y = kx^2$趋于原点,则$\lim_{x \to 0, y \to 0} \frac{x^2 y}{x^4 + y^2} = \lim_{x \to 0} \frac{kx^3}{x^4 + k^2 x^4} = \lim_{x \to 0} \frac{k}{1+k^2} = \frac{k}{1+k^2}$.当$k$取不同值时,即动点以不同方式趋于$(0,0)$时,函数有不同的极限,与极限存在的唯一性矛盾.
        故$\lim_{x \to 0, y \to 0} \frac{x^2 y}{x^4 + y^2} $不存在,从而$\begin{cases}
            \frac{x^2 y}{x^4 + y^2} ,  x^2 + y^2 \neq 0\\
            0, x^2 + y^2 = 0
        \end{cases}$在$(0,0)$处不连续;
        \item $\lim_{x \to 0, y \to 0} \left( 1 +xy \right)^{\frac{1}{x+y}} = \lim_{x \to 0, y \to 0} \left( 1 +xy \right)^{\frac{1}{x+y} \cdot \frac{x+y}{xy}} $.其中
        $$
        \lim_{x \to 0, y \to 0} (1+xy)^{\frac{1}{xy}} = \lim_{u \to 0} (1+u)^{\frac{1}{u}} = \e
        $$ 

        而取$y = -x + kx^2$时,$\lim_{x \to 0, y \to 0} \frac{x+y}{xy} = \lim_{x \to 0, y \to 0} \frac{x+y}{xy} = \lim_{x \to 0, y \to 0} \frac{-x^2 + kx^3}{kx^2} = -\frac{1}{k}$.
        即$k$取不同值时,$M(x,y)$沿$y = -x + kx^2$趋于原点时,函数有不同的极限,与极限存在的唯一性矛盾.
        $\lim_{x \to 0 ,y \to 0} \frac{xy}{x+y} $不存在,从而$\lim_{x \to 0, y \to 0} \left( 1 +xy \right)^{\frac{1}{x+y}} $不存在.
    \end{enumerate}
\end{proof}

\begin{example}
    设$f(x,y) = \begin{cases}
        \frac{x^2 y}{x^4 + y^2}, x^2 + y^2 \neq 0\\
        0, x^2 + y^2 = 0
    \end{cases}$.证明在$(0,0)$处过此点的每一条射线$\begin{cases}
        x = t \cos \alpha \\
        y = t \sin \alpha
    \end{cases}, 0 \les t < + \infty$,$f(x,y)$都连续,即$\lim_{t \to \delta^+} f(t \cos \alpha, t \sin \alpha) = f(0,0) = 0$.但$f(x,y)$在$(0,0)$处不连续.
\end{example}

\begin{proof}
    不连续性已在上文证明.下证射线上的连续性.

    $\lim_{t \to 0^+} f(t \cos \alpha, t \sin \alpha) = \lim_{t \to 0^+} \frac{(t \cos \alpha)^2 (t \sin \alpha)}{(t \cos \alpha)^4 + (t \sin \alpha)^3} = \lim_{t \to 0^+} \frac{t \cos^2 \alpha \sin \alpha}{t^2 \cos^4 \alpha + \sin^2 \alpha} = 0 = f(0,0)$.
    因此对于任意$\alpha$,即对于任意射线,函数$f(x,y)$在射线上连续.
\end{proof}

\section{连续多元函数的主要性质}

\begin{enumerate}
    \item 连续多元函数的和,差,积,商(分母不为零)仍然是连续的多元函数;
    \item 在复合有意义的前提下,连续多元函数的复合函数仍是连续函数;
    \item 有界闭区域$D$上的连续多元函数具有“五性”;
    \begin{enumerate}
        \item 有界性;
        \item 最值性;
        \item 介值性;
        \item 零值性;
        \item 一致连续性,
    \end{enumerate}
\end{enumerate}

上述性质的证明方法,与一元连续函数的“五性”证明方法类似.

\begin{homework}
    ex9.1:12,13,14(2)(7)(9)(10),15,17(1),18.
\end{homework}
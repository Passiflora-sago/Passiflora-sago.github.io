\chapter{数列极限的性质与应用}

\section{复习数列极限的线性性质}

\begin{theorem}[数列极限的线性性质]\label{thm:sequence_limit_linear}
    设$a,b,c_1,c_2$为常数且$\lim_{n \to \infty} a_n = a, \lim_{n \to \infty} b_n=b$,则 $$\lim_{n \to \infty}(c_1 a_n+c_2b_n)=c_1a+c_2b =c_1 \lim_{n \to \infty} a_n+c_2 \lim_{n \to \infty} b_n . $$
    
\end{theorem}

从上述极限的线性性质, 不难得到以下结论:

\begin{enumerate}
    \item 当$c_1=c_2=1$时, $\lim_{n \to \infty} (a_n+b_n)=\lim_{n \to \infty} a_n+\lim_{n \to \infty} b_n$;
    \item 当$c_1=1,c_2=-1$时, $\lim_{n \to \infty} (a_n-b_n)=\lim_{n \to \infty} a_n-\lim_{n \to \infty} b_n$;
    \item 当$c_1=k,c_2=0$时, $\lim_{n \to \infty} ka_n=k\lim_{n \to \infty} a_n$.
    \item 数列的线性性质可推广到任意有限个收敛数列的情形:设$a_{1n} \rightarrow a_1, a_{2n} \rightarrow a_2, \cdots, a_{mn} \rightarrow a_m$,且$a_1,a_2,\cdots,a_m,c_1,c_2,\cdots,c_m$为常数,则
    \begin{align*}
    & \lim_{n \to \infty}(c_1a_{1n}+c_2a_{2n}+\cdots+c_ma_{mn}) \\
    =& c_1a_1+c_2a_2+\cdots+c_ma_m \\
    =&c_1\lim_{n \to \infty} a_{1n}+c_2\lim_{n \to \infty} a_{2n}+\cdots+c_m\lim_{n \to \infty} a_{mn}
    \end{align*}

    对$\forall m\in N^*$成立.
\end{enumerate}

\section{数列极限的“四性”}

\begin{proposition}[数列极限的“四性”]\label{prop:sequence_limit_four_properties}
    设$\{a_n\}$收敛,则$\{a_n\}$满足以下四性:
    \begin{enumerate}
        \item 有界性:若$\{a_n\}$收敛,则$\{a_n\}$有界,反之未必;
        \item 唯一性:若$\{a_n\}$收敛,则其极限唯一;
        \item 保号性:若$\{a_n\}$收敛且$\lim_{n \to \infty} a_n = a$,$a_n \ges 0,\forall n \ges n_0$,则必有$a\ges 0$;
        \item 保序性:若$a_n \rightarrow a, b_n \rightarrow b$,且$a_n \les(\ges) b_n,\forall n \ges n_0$,则必有$a\les(\ges) b$.
    \end{enumerate}
\end{proposition}

\begin{proof}
    \begin{enumerate}
        \item 
        取 $\varepsilon = 1$, 由定义知道,当存在一个自然数 $N$, 使得当 $n > N$ 时,有 $|a_n - a| < 1$, 即当 $n > N$ 时,有 $|a_n| < |a| + 1$. 取
        $$
        M = \max(|a| + 1, |a_1|, |a_2|, \dots, |a_N|),
        $$
        注意到,第一,有有限个数中一定能取得一个最大的;第二,上面确定的 $M$ 显然与 $n$ 无关.则对所有自然数 $n$, 也就是说数列的所有项,都会有 $|a_n| \le M$.
        \item 
        如果 $\{a_n\}$ 有两个极限值 $a$ 和 $b$. 根据极限的定义,对于任意的正数 $\varepsilon$, 注意到 $\frac{\varepsilon}{2}$ 也是一个正数,因此对应两个极限值,分别存在正整数 $N_1$ 和 $N_2$, 使得当
        $$
        n > N_1 \text{ 时有 } |a_n - a| < \frac{\varepsilon}{2},
        $$
        $$
        n > N_2 \text{ 时有 } |a_n - b| < \frac{\varepsilon}{2}.
        $$
        因此,当 $n > \max(N_1, N_2)$ 时(即 $n > N_1, n > N_2$),上面两个不等式都满足,所以
        $$
        |a - b| = |(a - a_n) + (a_n - b)| \leq |a - a_n| + |a_n - b| < \frac{\varepsilon}{2} + \frac{\varepsilon}{2} = \varepsilon.
        $$
        两个数的距离要小于任意一个正数,这两个数必须相等,即 $a = b$.
        \item 
        若 $a > l$, 取 $\ve = a - l > 0$, 则存在一个自然数 $N$, 使得当 $n > N$ 时,有
        $$
        |a_n - a| < \ve = a - l, \text{因此} - (a - l) < a_n - a
        $$
        即,当 $n > N$ 时,不等式 $a_n > l$ 成立.对于 $a < l$ 的情况,可类似证明,在这种情况下,只要取 $\ve = l - a > 0$ 即可.对于此问,取$l = 0$.
        \item
        令$c_n = b_n - a_n$,则$c_n \rightarrow b - a$,且$c_n \les 0,\forall n \ges n_0$,由保号性可知,$b - a \les 0$,即$a \les b$.
    \end{enumerate}
\end{proof}

\begin{remark}
    其中唯一性暗示了,改变数列中有限多项的值,不会影响数列的收敛性及其极限.例如,对于数列 $1, 1/2, 1/3, 1/4, \dots$, 它的极限是 $0$, 即 $\lim_{n \to \infty} \frac{1}{n} = 0$. 如果我们改变数列的前 $10$项, 如$1,1,1,1,1,1,1,1,1,1,1/11,1/12,1/13,1/14,\dots$, 则数列的极限仍然是 $0$.这个性质在证明数列极限的存在性时,常常会被用到.

有界性质给出了收敛数列的一个必要条件.因此无界数列一定是发散的.例如对于数列 $0,1,0,2,0,3,0,4,\cdots$ 显然是无界的,且发散的.

保号性的条件是不严格不等,若调整为$a_n >0$, 则无法说明$a>0$.例如数列$1, 1/2, 1/3, 1/4, \dots$的极限是$0$,但数列的每一项都是正数.
\end{remark}




\section{收敛数列极限的四则运算法则}

\begin{theorem}[收敛数列极限的四则运算法则]\label{thm:sequence_limit_arithmetic}
    设$\{a_n\},\{b_n\}$收敛,且$\lim_{n \to \infty} a_n = a, \lim_{n \to \infty} b_n = b$,则有

    \begin{enumerate}
        \item $\lim\limits_{n \to \infty} (a_n \pm b_n) = \lim\limits_{n \to \infty} a_n \pm \lim\limits_{n \to \infty} b_n$.
        \item $\lim\limits_{n \to \infty} a_n b_n = \lim\limits_{n \to \infty} a_n \cdot \lim\limits_{n \to \infty} b_n$.
        \item $\lim\limits_{n \to \infty} \displaystyle \frac{a_n}{b_n} = \displaystyle \frac{\lim\limits_{n \to \infty} a_n}{\lim\limits_{n \to \infty} b_n}$, 其中 $\lim\limits_{n \to \infty} b_n \neq 0$.
    \end{enumerate}
    
\end{theorem}

\begin{proof}
    \begin{enumerate}
        \item 目的时要证明对于任意的正数 $\epsilon$, 能够找到一个正整数 $N$, 使得当 $n > N$ 时,不等式
        $$
        |a_n \pm b_n - (a \pm b)| \leq \epsilon
        $$
        成立.由 $\{a_n\}, \{b_n\}$ 的收敛性知,对于任意 $\frac{\epsilon}{2}$,分别存在正整数 $N_1$ 和 $N_2$ 使得
        $$
        \text{当 } n > N_1 \text{ 时有 } |a_n - a| < \frac{\epsilon}{2},
        $$
        以及
        $$
        \text{当 } n > N_2 \text{ 时有 } |b_n - b| < \frac{\epsilon}{2}.
        $$
        取 $N = \max(N_1, N_2)$,则当 $n > N$ 时,上面两个式子同时成立,因此有
        $$
        |a_n \pm b_n - (a \pm b)| \leq |a_n - a| + |b_n - b| < \frac{\epsilon}{2} + \frac{\epsilon}{2} = \epsilon.
        $$
        \item 注意到
        $$
        |a_n b_n - ab| \leq |a_n b_n - a_n b| + |a_n b - ab| = |a_n| |b_n - b| + |b_n - b| |a_n - a|.
        $$
        由于 $\{a_n\}, \{b_n\}$ 是收敛数列,故都是有界的,取一个大的界 $M$, 使得
        $$
        |a_n|, |b_n| < M (n \ges 1)
        $$
        因此 $|b| \leq M$ (定理 1.4 中的 3°).对于任意的正数 $\epsilon$, 对应 $\frac{\epsilon}{2M}$,分别存在整数 $N_1$ 和 $N_2$, 使得当 $n > N$ 时,
        $$
        |a_n - a| < \frac{\epsilon}{2M}, |b_n - b| < \frac{\epsilon}{2M}.
        $$
        同时成立.因此当 $n > N$ 时,有
        $$
        |a_n b_n - ab| < M |b_n - b| + M |a_n - a| < M \cdot \frac{\epsilon}{2M} + M \cdot \frac{\epsilon}{2M} = \epsilon.
        $$
        \item 因为
        $$
        \frac{a_n}{b_n} = a_n \cdot \frac{1}{b_n},
        $$
        且 $b \neq 0$, 由 2° 可知,只需证明数列 $\left\{\frac{1}{b_n}\right\}$ 收敛于 $\frac{1}{b}$ 即可.假设 $b > 0$,则
        $$
        \left| \frac{1}{b_n} - \frac{1}{b} \right| = \left| \frac{|b_n - b|}{|b_n b|} \right|.
        $$
        由于 $b_n$ 收敛于 $b$, 一方面对于正数 $b/2 > 0$, 存在 $N_1$, 当 $n > N_1$ 时,
        $$
        |b_n - b| < \frac{b}{2}.
        $$
        另一方面,对于任意给定的正数 $\epsilon$, 存在 $N_2$, 使得当 $n > N_2$ 时,
        $$
        |b_n - b| < \frac{b^2 \epsilon}{2}.
        $$
        所以,当 $n > N = \max(N_1, N_2)$ 时,
        $$
        \left| \frac{1}{b_n} - \frac{1}{b} \right| \leq |b_n - b| \cdot \frac{2}{b^2} \cdot \frac{\epsilon}{2} = \epsilon.
        $$
        即
        $$
        \lim_{n \to \infty} \frac{1}{b_n} = \frac{1}{b}.
        $$
        
    \end{enumerate}
    
\end{proof}

定理\ref{prop:sequence_limit_four_properties}说明收敛数列的极限运算和四则运算是可以交换的,并可推广到有限多个收敛数列与四则运算的情况.对于 3 中的结论,会因为某些 $b_n$ 为 $0$ 而使得分式没有意义. 但是因为 $\{b_n\}$ 的极限 $b \neq 0$,所以 $b_n$ 为 $0$ 的项至多只有有限个.可以改变这有限多项的值,这不会改变 $\{b_n\}$ 的收敛性和极限.或者在 $\{a_n b_n\}$ 中删去这些没有定义的有限多项,不会改变其收敛性和极限.
有了定理\ref{prop:sequence_limit_four_properties},在计算复数列极限时,可以将其化为简极限的四则运算,而不必再使用“$\ve$-$N$”语言作繁琐的论述.


\section{例题}

\begin{example}\label{example:e_n}
    设$a_n= (1+ \frac 1n )^n , n \in N^*$,证明:
    \begin{enumerate}
        \item $\lim_{n \to \infty} a_n = \e \approx 2.7182818128$;
        \item $\underset{x\rightarrow +\infty }{\lim} (1 +\frac 1x )^x = \e =\underset{x\rightarrow -\infty }{\lim} (1 +\frac 1x )^x,x \in R$
        \item $\underset{x\rightarrow \infty }{\lim} (1 +\frac 1x )^x = \e, x \in R$
    \end{enumerate}

    \begin{solution}
        函数极限还没有讲到, 此处仅证明第1问.证明数列$a_n$收敛即可.
        首先证明该数列是递增的。事实上,由二项式定理可得
        $$
        a_n = 1 + \sum_{k=1}^{n} C_n^k \cdot \frac{1}{n^k} = 1 + \sum_{k=1}^{n} \frac{1}{k!}.
        $$
        $$
        = 1 + 1 + \sum_{k=2}^{n} \frac{1}{k!} \left( 1 - \frac{1}{n} \right) \left( 1 - \frac{2}{n} \right) \dots \left( 1 - \frac{k-1}{n} \right),
        $$
        $$
        a_{n+1} = 1 + 1 + \sum_{k=2}^{n} \frac{1}{k!} \left( 1 - \frac{1}{n+1} \right) \left( 1 - \frac{2}{n+1} \right) \dots \left( 1 - \frac{k-1}{n+1} \right) + \left( \frac{1}{n+1} \right)^{n+1}.
        $$
        比较 $a_n$ 和 $a_{n+1}$ 两个表达式的右端和号中的对应项,显然,前者较小。而 $a_{n+1}$ 所多出来的一项 $\left( \frac{1}{n+1} \right)^{n+1} > 0$,故 $a_{n+1} > a_n$。所以 $\{a_n\}$ 为严格递增数列。
        
        其次,我们将证明数列是有界的。在 $a_n$ 的上述展开式中,
        $$
        0 < \left( 1 - \frac{1}{n} \right) \left( 1 - \frac{2}{n} \right) \dots \left( 1 - \frac{k-1}{n} \right) < 1.
        $$
        所以
        $$
        2 < a_n < 2 + \sum_{k=2}^{n} \frac{1}{k!} = 1 + \frac{1}{1!} + \frac{1}{2!} + \cdots + \frac{1}{n!}.
        $$
        $$
        < 2 + \frac{1}{1 \cdot 2} + \frac{1}{2 \cdot 3} + \cdots + \frac{1}{n(n-1)} = 3 - \frac{1}{n} < 3,
        $$
        即 $n = 2, 3, \dots$,也就是说数列 $\{a_n\}$ 是单调递增且有上界的,因此一定收敛。
    \end{solution}

\end{example}



\begin{example}
    证明闭区间套定理.

    \begin{theorem}[闭区间套定理]
        若$\{[a_n,b_n]\}$是一列闭区间,满足$[a_n,b_n]\supset[a_{n+1},b_{n+1}],n=1,2,\cdots$,且$\lim_{n \to \infty} (b_n-a_n)=0$,则存在唯一的实数$\xi$,使得$\xi \in [a_n,b_n],n=1,2,\cdots$.
    \end{theorem}

    所有区间的左端点构成的数列$\{a_n\}$是单调递增有上界的,故有极限,记为$a$,即$\lim_{n \to \infty} a_n = a$.同理,所有区间的右端点构成的数列$\{b_n\}$是单调递减有下界的,故有极限,记为$b$,即$\lim_{n \to \infty} b_n = b$.因此 $a-b = \lim_{n \to \infty} (a_n - b_n) = 0$,即$a=b$.
    即证存在$\xi = a = b$.

    若存在另一实数$\eta \in [a_n,b_n],n=1,2,\cdots$,则$\xi \les \eta \les \xi$,即$\xi = \eta$.故唯一性得证.
\end{example}

\begin{remark}
    闭区间套定理,是刻画实数集$\R$的连续性的五个等价公理之一.
\end{remark}

为了纪念数学家Euler(欧拉)在其中的贡献,将$\lim_{n \to \infty} (1+\frac 1n )^n $记为 $\e$.经计算可知,$\e \approx 2.718281828$.讲义中还证明了 $\e$是一个无理数,且将以 $\e$为底的对数称为自然对数,记为$\ln x$,即$\ln x = \log_\e x$.

\begin{homework}
ex1.2: 14,15(3)(4),16,18(3),22(2)(4);CH1:3(2).
\end{homework}

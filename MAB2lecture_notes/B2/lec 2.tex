\setcounter{chapter}{1}
\chapter{空间平面与直线}

\section{平面(plane)的五种表示形式}

\begin{definition}[平面的五种表示形式]
    \begin{enumerate}
        \item 向量式:设平面$\pi$过已知点$M_0(x_0,y_0,z_0)$,且与已知的非零向量$\n=(A,B,C)$垂直,则平面$\pi$唯一确定.设$P(x,y,z)$为$\pi$上任一点,则有$\overrightarrow{M_0P}\bot\n$,于是有$$\n\cdot \overrightarrow{M_0P}=0$$称为平面$\pi$的向量式方程.
        \item 点法式:由向量式,有$\n\cdot \overrightarrow{M_0P}=0$,即$$A(x-x_0)+B(y-y_0)+C(z-z_0)=0$$,称为平面$\pi$的向量式方程.
        \item 一般式:设$D=-Ax_0-By_0-Cz_0$,则有$$Ax+By+Cz+D=0$$称为平面$\pi$的一般式方程.
        \item 截距式:一般式中,设$d\triangleq-D\neq 0$,令$\frac{d}{A}=a,\frac{d}{B}=b,\frac{d}{C}=c$,则$$\frac{x}{a}+\frac{y}{b}+\frac{z}{c}=1$$称为平面$\pi$的截距式方程.
        \item 三点式:设$A(x_1,y_1,z_1),B(x_2,y_2,z_2),C(x_3,y_3,z_3)$为$\pi$上不共线的三点,则由$A,B,C$三点确定唯一的平面$\pi$,设$P(x,y,z)$为$\pi$上任一点,则有$\overrightarrow{AP},\overrightarrow{AB},\overrightarrow{AC}$共面,即$$\begin{vmatrix}x-x_1&y-y_1&z-z_1\\x_2-x_1&y_2-y_1&z_2-z_1\\x_3-x_1&y_3-y_1&z_3-z_1\end{vmatrix}=0$$称为平面$\pi$的三点式方程.
    \end{enumerate}
\end{definition}

\section{空间直线(line)的五种表示形式}
\begin{definition}[空间直线的五种表示形式]
    \begin{enumerate}
        \item 向量式:设直线$l$过已知点$M_0(x_0,y_0,z_0)$,且与已知的非零向量$\btau=(l,m,n)$平行,则直线$l$唯一确定.设$P(x,y,z)$为$l$上任一点,则有$\overrightarrow{M_0P}\parallel\n$,于是有$$\overrightarrow{M_0P}\times\btau=\0$$称为直线$l$的向量式方程.
        \item 点向式:由向量式,有$\overrightarrow{M_0P}\parallel\btau$,则有$$\frac{x-x_0}{l}=\frac{y-y_0}{m}=\frac{z-z_0}{n}$$,称为直线$l$的点向式方程.
        \item 参数式,在点向式中,令$\frac{x-x_0}{l}=\frac{y-y_0}{m}=\frac{z-z_0}{n}=t$,则有$$\begin{cases}x=x_0+lt\\y=y_0+mt\\z=z_0+nt\end{cases}$$称为直线$l$的参数式方程.
        \item 交面式:设平面$p_1:A_1x+B_1y+C_1z+D_1=0$和平面$p_2:A_2x+B_2y+C_2z+D_2=0$不平行,则$\pi_1$与$\pi_2$有交线$l$,$l$上的点$P(x,y,z)$满足$$\begin{cases}A_1x+B_1y+C_1z+D_1=0\\A_2x+B_2y+C_2z+D_2=0\end{cases}$$称为直线$l$的交面式方程.
        \item 两点式:设$Q_1(x_1,y_1,z_1),Q_2(x_2,y_2,z_2)$为直线$l$上的两点,则由$Q_1,Q_2$确定唯一的直线$l$,设$P(x,y,z)$为$l$上任一点,则有$\overrightarrow{Q_1P},\overrightarrow{Q_1Q_2}$共线,即$$\frac{x-x_1}{x_2-x_1}=\frac{y-y_1}{y_2-y_1}=\frac{z-z_1}{z_2-z_1}$$称为直线$l$的两点式方程.   
    \end{enumerate}
\end{definition}
\newpage
\section{面面,线线,线面之间的关系}
\begin{proposition}
    设$$\begin{cases}
        \pi_1:A_1x+B_1y+C_1z+D_1=0,&\n_1=(A_1,B_1,C_1)\\
        \pi_2:A_2x+B_2y+C_2z+D_2=0,&\n_2=(A_2,B_2,C_2)\\
        L_1:\frac{x-x_1}{l_1}=\frac{y-y_1}{m_1}=\frac{z-z_1}{n_1},&\btau_1=(l_1,m_1,n_1)\\
        L_2:\frac{x-x_2}{l_2}=\frac{y-y_2}{m_2}=\frac{z-z_2}{n_2},&\btau_2=(l_2,m_2,n_2)
    \end{cases}$$
    则\begin{enumerate}
        \item $\pi_1\parallel\pi_2\Leftrightarrow\frac{A_1}{A_2}=\frac{B_1}{B_2}=\frac{C_1}{C_2}\Leftrightarrow\n_1\times\n_2=\0$.
        \item $\pi_1\bot\pi_2\Leftrightarrow A_1A_2+B_1B_2+C_1C_2=0\Leftrightarrow\n_1\cdot\n_2=0$.
        \item $\pi_1$与$\pi_2$的夹角$\alpha(0<\alpha\les\pi),\cos\alpha=\frac{\n_1\cdot\n_2}{\left|\n_1\right|\cdot\left|\n_2\right|}=\n_1^0\cdot\n_2^0$,即得$\alpha=\arccos\frac{\n_1\cdot\n_2}{\left|\n_1\right|\cdot\left|\n_2\right|}$.
        \item $L_1\parallel L_2\Leftrightarrow\btau\parallel\btau_2\Leftrightarrow\btau_1\times\btau_2=\0\Leftrightarrow\frac{l_1}{l_2}=\frac{m_1}{m_2}=\frac{n_1}{n_2}$.
        \item $L_1\bot L_2\Leftrightarrow\btau_1\bot\btau_2\Leftrightarrow\btau_1\cdot\btau_2=0\Leftrightarrow l_1l_2+m_1m_2+n_1n_2=0$.
        \item $L_1$与$L_2$的夹角$\alpha(0\les\alpha\les\pi),\cos\alpha=\frac{\btau_1\cdot\btau_2}{\left|\btau_1\right|\cdot\left|\btau_2\right|}=\btau_1^0\cdot\btau_2^0$,即得$\alpha=\arccos\frac{\btau_1\cdot\btau_2}{\left|\btau_1\right|\cdot\left|\btau_2\right|}$.
        \item $L_1\bot\pi_1\Leftrightarrow\btau_1\parallel\n_1\Leftrightarrow\btau_1\times\n_1=\0\Leftrightarrow \frac{A_1}{l_1}=\frac{B_1}{m_1}=\frac{C_1}{n_1}$.
        \item $L_1\parallel\pi_1\Leftrightarrow\btau_1\bot\n_1\Leftrightarrow\btau_1\cdot\n_1=0\Leftrightarrow A_1l_1+B_1m_1+C_1n_1=0$.
        \item $L_1$与$\pi_1$的夹角$\beta(0\les\beta\les\frac{\pi}{2}),\cos(\frac{\pi}{2}-\beta)=\left|\frac{\btau_1\cdot\n_1}{\left|\btau_1\right|\cdot\left|\n_1\right|}\right|=\left|\btau_1^0\cdot\n_1^0\right|$,故$\sin\beta=\left|\btau_1^0\cdot\n_1^0\right|0$,即得$\beta=\arcsin\left|\btau_1^0\cdot\n_1^0\right|$.
    \end{enumerate}
\end{proposition}

\section{例题}
\begin{example}
    分别求已知点$M(1,-1,-2)$关于点$A(1,0,1)$,平面$\pi:3x+4y-5z-1=0$,直线$L:\frac{x-1}{-1}=\frac{y-1}{2}=\frac{z-1}{0}$的对称点$Q(x,y,z)$.
\end{example}
\begin{solution}
    \begin{enumerate}
        \item $A(1,0,1)$是线段$MQ$的中点,有$\begin{cases}\frac{1+x}{2}=1\\y=\frac{-1+y}{2}=0\\z=\frac{-2+z}{2}=1\end{cases}\Rightarrow\begin{cases}x=1\\y=1\\z=4\end{cases}$,即$Q(1,1,4)$.
        \item $\n=(3,4,-5)$,则有$\begin{cases}\overrightarrow{MQ}\parallel\n\\MQ\text{中点}N(\frac{x+1}{2},\frac{y-1}{2},\frac{z-2}{2})\text{在}\pi\text{上} \end{cases}$
        
        $\Rightarrow\begin{cases}x=1+3t\\y=-1+4t\\z=-2-5t\\3(\frac{x+1}{2})+4(\frac{y-1}{2})-5\frac{z-2}{2}-1=0\end{cases}$
        
        $\Rightarrow t=\frac{12}{25}$,带回$x,y,z$可知$Q(\frac{61}{25},\frac{23}{25},-\frac{10}{25})$.

        \item $\btau=(-1,2,0)$,则有$\begin{cases}\overrightarrow{MQ}\bot\btau\\MQ\text{中点}N(\frac{x+1}{2},\frac{y-1}{2},\frac{z-2}{2})\text{在}L\text{上} \end{cases}$
        
        $\Rightarrow\begin{cases}\frac{x+1}{2}=1-t\\\frac{y-1}{2}=1+2t\\\frac{z-2}{2}=1\\-1(x-1)+2(y+1)+0(z+2)=0\end{cases}$

        $\Rightarrow \begin{cases}x=1-2t\\\y=3+4t\\\z=4\\-1(x-1)+2(y+1)+0(z+2)=0\end{cases}$

        $\Rightarrow t=-\frac{5}{3}$,带回$x,y,z$可知$Q(\frac{13}{3},-\frac{1}{5},4)$.
    \end{enumerate}
\end{solution}

\begin{example}
    设$A(1,0,1),B(0,1,1),C(2,0,3),D(1,1,1)$为已知的四点,求
    \begin{enumerate}
        \item 求四面体$\Omega:A-BCD$的体积$V(\Omega)$.
        \item 求$B,C,D$三点确定的三角形$\triangle$的面积$S_\triangle$.
        \item 求$B,C,D$三点确定的平面方程.
    \end{enumerate}
\end{example}
\begin{solution}
    \begin{enumerate}
        \item $V(\Omega)=\frac{1}{6}\left|(\overrightarrow{BC}\times\overrightarrow{BD})\cdot\overrightarrow{BA}\right|=\frac{1}{6}\left|\begin{vmatrix}2&-1&2\\1&0&0\\1&-1&0\end{vmatrix}\right|=\frac{1}{6}\times2=\frac{1}{3}$.
        \item $S_\triangle=\frac{1}{2}\left|\overrightarrow{BC}\times\overrightarrow{BD}\right|=\frac{1}{2}\left|\begin{vmatrix}\i&\j&\k\\2&-1&2\\1&0&0\end{vmatrix}\right|=\frac{1}{2}\left|0\i+2\j+1\k\right|=\frac{1}{2}\sqrt{0^2+2^2+1^2}=\frac{\sqrt{5}}{2}$.
        \item 设$P(x,y,z)$为$\pi$中的任一点,则$\overrightarrow{BP},\overrightarrow{BC},\overrightarrow{BD}$共面,即$\begin{vmatrix}x-0&y-1&z-1\\2&-1&2\\1&0&0\end{vmatrix}=0$,解得$\pi:2y+z-3=0$为所求平面方程.
    \end{enumerate}
\end{solution}

\begin{homework} 
    ex8.2:1,2,3,6,7,14(1),15(1),16.
\end{homework}
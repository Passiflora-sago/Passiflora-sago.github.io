\setcounter{chapter}{15} % 设置章节计数器

\chapter{多元函数微分学复习小结}\

\section{多元函数与一元函数}

多元函数微分学相较于一元函数微分学,不同之处在于多元函数的极限动点区域定点有无穷多种方式,而一元函数的极限只有左右极限两种方式.
其他不同的性质基本由上述性质引申而来.

\begin{example}
    设$z = f(x,y)$在区域$D$中偏导数存在且有界,即$\exists M > 0$,使得$\forall (x,y) \in D, |f_x'(x,y)| \les M, |f_y'(x,y)| \les M$,则$f$在$D$中是连续的.
\end{example}

\begin{proof}
    $\forall M_0(x_0,y_0) \in D$,设$M(x_0+ \Delta x,y_0+\Delta y) \in D$,则
    \begin{align*}
        \Delta z = &f(M) - f(M_0) = f(x_0 + \Delta x,y_0 + \Delta y) - f(x_0,y_0)\\
        = &\left( f(x_0 + \Delta x,y_0 + \Delta y) - f(x_0 , y_0 + \Delta y) \right) + \left( f(x_0 , y_0 + \Delta y) - f(x_0,y_0) \right)\\
        =& f_x'(x_0 +\theta_1 \Delta x,y_0 + \Delta y) \Delta x + f_y'(x_0,y_0 + \theta_2 \Delta y) \Delta y\\
        \les& M \left( |\Delta x| + |\Delta y| \right)\\
    \end{align*}
    当$\Delta x, \Delta y \to 0$时,有$\Delta z \to 0$,即$f$在$M_0$处连续.
\end{proof}

\section{例题}

\begin{example}
    证明:一切二次曲面$\Sigma$
    $$\Sigma: ax^2 + by^2 + cz^2 + dx + ey + fz + g = 0$$
    在点$M_0(x_0,y_0,z_0)$处的切平面方程为
    $$ax_0x + by_0y + cz_0z + d\frac{x_0+x}{2} + e\frac{y_0+y}{2} + f\frac{z_0+z}{2} + g = 0$$
    其中$a,b,c,d,e,f,g$为常数,且$a^2+b^2+c^2>0$.
\end{example}

\begin{proof}
    令$F(x,y,z) = ax^2 + by^2 + cz^2 + dx + ey + fz + g$,则$F$在$M_0$处的切平面的法向量为
    $$\n = \nabla F \bigg|_{M_0} = \left( F_x',F_y',F_z' \right) \bigg|_{M_0} = \left( 2ax_0 + d, 2by_0 + e, 2cz_0 + f \right)$$
    因此由点法式方程,有切平面方程为
    $$(2ax_0 +d)(x-x_0) + (2by_0 + e)(y-y_0) + (2cz_0 + f)(z-z_0) = 0$$
    整理后可得
    $$ax_0x + by_0y + cz_0z + d\frac{x_0+x}{2} + e\frac{y_0+y}{2} + f\frac{z_0+z}{2} + g = 0$$
\end{proof}


\begin{example}
    设有函数$z = 1 - \left( \frac{x^2}{a^2} + \frac{y^2}{b^2} \right)$,曲线
    $$L: \frac{x^2}{a^2} + \frac{y^2}{b^2} = 1$$
    其中$M_0 \left(\frac{a}{\sqrt{2}},\frac{b}{\sqrt{2}} \right)$
    为$L$上的点,记$\n$为$L$在$M_0$处的法向量,$\l$为任意方向向量.
    
    求$\pdv{z}{\n} \bigg|_{M_0}$,并求$\left( \pdv{z}{\l} \bigg|_{M_0} \right)_{\max} , \left( \pdv{z}{\l} \bigg|_{M_0} \right)_{\min}$.
\end{example}

\begin{solution}
    设$F(x,y) = \frac{x^2}{a^2} + \frac{y^2}{b^2} - 1$,则$F$在$M_0$处的法向量为
    $$\bm N = \nabla F \bigg|_{M_0} = \left( F_x',F_y' \right) \bigg|_{M_0} = \frac{\sqrt 2}{ab} \left( b,a \right) $$
    取$\bm N = (b,a)$,则$L$在$M_0$处的内法向为$\n = -bm N = (-b,-a)$,则方向向量为$\n^\circ = \left( -\frac{b}{\sqrt{a^2 + b^2}}, -\frac{a}{\sqrt{a^2 + b^2}} \right)$.又
    $$\nabla z \bigg|_{M_0} = \left( -\frac{2x}{a^2}, -\frac{2y}{b^2} \right) \bigg|_{M_0} = \left( -\frac{\sqrt 2}{a}, -\frac{\sqrt 2}{b} \right)$$
    且
    $$\left( \pdv{z}{\l} \bigg|_{M_0} \right)_{\max} = \left| \nabla z \right|_{M_0} = \left| \left(-\frac{\sqrt 2}{a}, -\frac{\sqrt 2}{b} \right) \right| = \sqrt{\frac{2}{a^2} + \frac{2}{b^2}}$$

    $$\left( \pdv{z}{\l} \bigg|_{M_0} \right)_{\min} = \left| \nabla z \cdot \n^\circ \right|_{M_0} = \left| \left(-\frac{\sqrt 2}{a}, -\frac{\sqrt 2}{b} \right) \right| = \sqrt{\frac{2}{a^2} + \frac{2}{b^2}}$$
\end{solution}

\begin{example}
    在椭球面$\Sigma: \frac{x^2}{a^2} + \frac{y^2}{b^2} + \frac{z^2}{c^2} = 1$上取一点$M_0(x_0,y_0,z_0)$,求$M_0$处的切平面方程与三个坐标面围成的四面体$\Omega$的体积的最小值.
\end{example}

\begin{solution}
    设$M_0$位于第一卦限,则$\pi$为$$\frac{x_0x}{a^2} + \frac{y_0y}{b^2} + \frac{z_0z}{c^2} = 1$$
    则$V(\Omega) = \frac16 \left( \frac{a^2}{x_0} \right) \left( \frac{b^2}{y_0} \right) \left( \frac{c^2}{z_0} \right) = \frac{abc}{6} \frac{1}{\sqrt{\left(\frac{x_0}{a}\right)^2 + \left(\frac{y_0}{b}\right)^2 + \left(\frac{z_0}{c}\right)^2}}$
    再利用平均值不等式,得到
    $$\frac{abc}{6} \frac{1}{\sqrt{\left(\frac{x_0}{a}\right)^2 + \left(\frac{y_0}{b}\right)^2 + \left(\frac{z_0}{c}\right)^2}} \ges \frac{abc}{6} \frac{1}{\sqrt{3}}$$
    故$V(\Omega)$的最小值为$\frac{abc}{6\sqrt{3}}$.此时切平面方程为$\frac{x}{a} + \frac{y}{b} + \frac{z}{c} = \sqrt{3}$.
\end{solution}


\begin{example}
    设$(r_0,\theta_0,\varphi_0)$为点$M_0(x_0,y_0,z_0)$的球坐标系,即
    $$\begin{cases}
        x_0 = r_0 \sin \theta_0 \cos \varphi_0,\\
        y_0 = r_0 \sin \theta_0 \sin \varphi_0,\\
        z_0 = r_0 \cos \theta_0.
    \end{cases}$$
    其中$r_0 \in [0,+\infty)$, $\theta_0 \in [0,\pi]$, $\varphi_0 \in [0,2\pi)$.证明:
    \begin{enumerate}
        \item 三个球坐标曲面$\Sigma_1: r = r_0$, $\Sigma_2: \theta = \theta_0$, $\Sigma_3: \varphi = \varphi_0$在$M_0$处两两正交
        \item 三条球坐标曲线$\Gamma_1: \begin{cases}
            \Sigma_1,\\
            \Sigma_2
        \end{cases}$, $\Gamma_2: \begin{cases}
            \Sigma_1,\\
            \Sigma_3
        \end{cases}$, $\Gamma_3: \begin{cases}
            \Sigma_2,\\
            \Sigma_3
        \end{cases}$在$M_0$处两两正交
    \end{enumerate}
\end{example}

\begin{proof}
    \begin{enumerate}
        \item 实际上要验证的是,三张曲面的法向量在$M_0$处两两正交.
        $\Sigma_1$在球坐标系下的方程为$r = r_0$,则在直角坐标系下的方程为
        $$\sqrt{x^2 + y^2 + z^2} = r_0$$
        设$F_1(x,y,z) = x^2 + y^2 + z^2 - r_0^2$,则$F_1$在$M_0$处的法向量为
        $$\n_1 = \nabla F_1 \bigg|_{M_0} = \left( 2x_0, 2y_0, 2z_0 \right)$$
        $\Sigma_2$在球坐标系下的方程为$\theta = \theta_0$,则在直角坐标系下的方程为
        $$\frac{z}{\sqrt{x^2+y^2}} = \cos \theta_0$$
        设$F_2(x,y,z) = z^2 - (x^2 + y^2) \cos^2 \theta_0$,则$F_2$在$M_0$处的法向量为
        $$\n_2 = \nabla F_2 \bigg|_{M_0} = \left( -2x_0 \cos^2 \theta_0, -2y_0 \cos^2 \theta_0, 2z_0 \right)$$
        $\Sigma_3$在球坐标系下的方程为$\varphi = \varphi_0$,则在直角坐标系下的方程为
        $$\frac{y}{x} = \tan \varphi_0$$
        设$F_3(x,y,z) = y - x \tan \varphi_0$,则$F_3$在$M_0$处的法向量为
        $$\n_3 = \nabla F_3 \bigg|_{M_0} = \left( -\tan \varphi_0, 1, 0 \right)$$

        而
        \begin{align*}
            \n_1 \cdot \n_2 &= \left( 2x_0, 2y_0, 2z_0 \right) \cdot \left( -2x_0 \cos^2 \theta_0, -2y_0 \cos^2 \theta_0, 2z_0 \right)\\
            &= -4x_0^2 \cos^2 \theta_0 - 4y_0^2 \cos^2 \theta_0 + 4z_0^2 = 0\\
            \n_1 \cdot \n_3 &= \left( 2x_0, 2y_0, 2z_0 \right) \cdot \left( -\tan \varphi_0, 1, 0 \right)\\
            &= -2x_0 \tan \varphi_0 + 2y_0 = 2\left( y_0 - x_0 \tan \varphi_0 \right) = 0\\
            \n_2 \cdot \n_3 &= \left( -2x_0 \cos^2 \theta_0, -2y_0 \cos^2 \theta_0, 2z_0 \right) \cdot \left( -\tan \varphi_0, 1, 0 \right)\\
            &= 2x_0 \cos^2 \theta_0 \tan \varphi_0 - 2y_0 \cos^2 \theta_0 = 2\left( x_0 \cos^2 \theta_0 \tan \varphi_0 - y_0 \cos^2 \theta_0 \right)\\
            &= 2\left( x_0 \cos^2 \theta_0 \tan \varphi_0 - y_0 \cos^2 \theta_0 \right) = 0
        \end{align*}
        \item 要验证的是,三条曲线的切向量$\tau_1,\tau_2,\tau_3$在$M_0$处两两正交.
        其中$$\tau_1 \parallel \n_1 \times \n_2$$
        因此又$\n_1,\n_2,\n_3$两两正交,所以
        $$\tau_1 \parallel \n_3$$
        同理可得$\tau_2 \parallel \n_2$,$\tau_3 \parallel \n_1$,所以$\tau_1,\tau_2,\tau_3$两两正交.
    \end{enumerate}
\end{proof}

\begin{example}
    设$(r_0,\theta_0,z_0)$为点$M_0(x_0,y_0,z_0)$的圆柱坐标系,即
    $$\begin{cases}
        x_0 = r_0 \cos \theta_0,\\
        y_0 = r_0 \sin \theta_0,\\
        z_0 = z_0.
    \end{cases}$$
    证明:
    \begin{enumerate}
        \item 三个圆柱坐标曲面$\Sigma_1: r = r_0$, $\Sigma_2: \theta = \theta_0$, $\Sigma_3: z = z_0$在$M_0$处两两正交
        \item 三条圆柱坐标曲线$\Gamma_1: \begin{cases}
            \Sigma_1,\\
            \Sigma_2
        \end{cases}$, $\Gamma_2: \begin{cases}
            \Sigma_1,\\
            \Sigma_3
        \end{cases}$, $\Gamma_3: \begin{cases}
            \Sigma_2,\\
            \Sigma_3
        \end{cases}$在$M_0$处两两正交
    \end{enumerate}
\end{example}

\begin{proof}
    \begin{enumerate}
        \item 实际上要验证的是,三张曲面的法向量在$M_0$处两两正交.
        $\Sigma_1$在圆柱坐标系下的方程为$r = r_0$,则在直角坐标系下的方程为
        $$\sqrt{x^2 + y^2} = r_0$$
        设$F_1(x,y,z) = x^2 + y^2 - r_0^2$,则$F_1$在$M_0$处的法向量为
        $$\n_1 = \nabla F_1 \bigg|_{M_0} = \left( 2x_0, 2y_0, 0 \right)$$
        $\Sigma_2$在圆柱坐标系下的方程为$\theta = \theta_0$,则在直角坐标系下的方程为
        $$\frac{y}{x} = \tan \theta_0$$
        设$F_2(x,y,z) = y - x \tan \theta_0$,则$F_2$在$M_0$处的法向量为
        $$\n_2 = \nabla F_2 \bigg|_{M_0} = \left( -\tan \theta_0, 1, 0 \right)$$
        $\Sigma_3$在圆柱坐标系下的方程为$z = z_0$,则在直角坐标系下的方程为
        $$z = z_0$$
        设$F_3(x,y,z) = z - z_0$,则$F_3$在$M_0$处的法向量为
        $$\n_3 = \nabla F_3 \bigg|_{M_0} = \left( 0, 0, 1 \right)$$

        而
        \begin{align*}
            \n_1 \cdot \n_2 &= \left( 2x_0, 2y_0, 0 \right) \cdot \left( -\tan \theta_0, 1, 0 \right)\\
            &= -2x_0 \tan \theta_0 + 2y_0 = 2\left( y_0 - x_0 \tan \theta_0 \right) = 0\\
            \n_1 \cdot \n_3 &= \left( 2x_0, 2y_0, 0 \right) \cdot (0, 0, 1) = 0\\
            \n_2 \cdot \n_3 &= (-\tan \theta_0, 1, 0) \cdot (0, 0, 1) = 0
        \end{align*}

        故三张曲面在$M_0$处两两正交.
        \item 曲线的正交同上述球坐标系中的正交.
    \end{enumerate}
\end{proof}

\begin{example}
    设$z = z(x,y)$是由方程$z^3 -2xz + y = 0$确定的隐函数,且$x=1,y=1$时$z = 1$.试将$z(x,y)$在点$M_0(1,1)$处展开为二阶Taylor公式.
\end{example}

\begin{solution}
    $z(x,y)$在$M_0$处的二阶Taylor公式为
    \begin{align*}
        z(x,y) &= z(1,1) + (x-1)z_x'(1,1) + (y-1)z_y'(1,1)\\
        &+ \frac{(x-1)^2}{2}z_{xx}''(1,1) + (x-1)(y-1)z_{xy}''(1,1) + \frac{(y-1)^2}{2}z_{yy}''(1,1) + o(\rho^2)\\
    \end{align*}

    对$z^3 -2xz + y = 0$两边求微分,得
    $$3z^2 \dif z - 2 \left( x \dif z + z \dif x\right) + \dif y = 0$$
    整理得
    $$\dif z = \frac{2z}{3z^2 - 2x} \dif x + \frac{1}{3z^2 - 2x} \dif y$$
    因此
    $$z_x' = \frac{2z}{3z^2 - 2x}, z_y' = \frac{1}{3z^2 - 2x}$$
    对于$z_x',z_y'$求偏导数,得
    $$z_{xx}'' = \frac{1}{(3z^2 - 2x)^2} \left( 2z_x'(3z^2-2x) - 2z (6zz_x' - 2)\right)$$
    $$z_{xy}'' = \frac{1}{(3z^2 - 2x)^2} \left( 2z_y'(3z^2-2x) - 2z (6zz_y') \right)$$
    $$z_{yy}'' = \frac{1}{(3z^2 - 2x)^2} \left(-(6zz_y')\right)$$

    代入$x=1,y=1,z=1$得
    $$ z_x' = 2, z_y' = -1$$
    $$z_{xx}'' = 16, z_{xy}'' = 10, z_{yy}'' = -6$$

    故$z(x,y)$在$M_0(1,1)$处的二阶Taylor公式为
    \begin{align*}
        z(x,y) &= 1 + 2(x-1) - (y-1) + \frac{16}{2}(x-1)^2 + 10(x-1)(y-1) - \frac{6}{2}(y-1)^2 + o(\rho^2)\\
        &= 1 + 2(x-1) - (y-1) + 8(x-1)^2 + 5(x-1)(y-1) - 3(y-1)^2 + o(\rho^2)
    \end{align*}
\end{solution}

\begin{example}
    求旋转椭球面$\Sigma:\frac{x^2}{4} + y^2 + z^2 = 1$上距平面$\pi:x+y+2z = 9$最远和最近的点.
\end{example}

\begin{solution}
    这是一个条件极值问题,取$M_0(x_0,y_0,z_0) \in \Sigma$,则$M_0$到$\pi$的距离$d = \frac{|x_0 + y_0 + 2z_0 - 9|}{\sqrt{1^2 + 1^2 + 2^2}} = \frac{|x_0 + y_0 + 2z_0 - 9|}{\sqrt{6}}$.取目标函数
    $$
    f(x,y,z) = (x+y+2z-9)^2
    $$
    条件为
$$g(x,y,z) = \frac{x^2}{4} + y^2 + z^2 - 1 = 0$$
作
$L(x,y,z,\lambda) = f(x,y,z) + \lambda g(x,y,z)$,则
$$ \begin{cases}
    L_x' = 2(x+y+2z-9) + \lambda \frac{x}{2} = 0\\
    L_y' = 2(x+y+2z-9) + \lambda 2y = 0\\
    L_z' = 4(x+y+2z-9) + \lambda 2z = 0\\
    L_\lambda' = \frac{x^2}{4} + y^2 + z^2 - 1 = 0
\end{cases} \Rightarrow \begin{cases}
    x_0 = \pm \frac{4}{3}\\
    y_0 = \pm \frac{1}{3}\\
    z_0 = \pm \frac{2}{3}\\
\end{cases}$$

$M_1(\frac{4}{3},\frac{1}{3},\frac{2}{3})$为最近点,$d= \sqrt 6$,

$M_2(-\frac{4}{3},-\frac{1}{3},-\frac{2}{3})$为最远点,$d = 2\sqrt 6 $.
\end{solution}

\begin{homework}
    ex9.5:5,7(5),8,10(3),11(2),16,17,19.
\end{homework}






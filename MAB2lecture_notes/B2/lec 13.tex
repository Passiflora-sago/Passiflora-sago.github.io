\setcounter{chapter}{12} % 设置章节计数器

\chapter{多元函数的Taylor公式及其应用}

\section{二元函数的Taylor公式}

设$D \subset \R^2$,且$D$是凸区域,即
$$
\forall x, y \in D, \lambda x+(1-\lambda) y \in D, \forall \lambda \in[0,1]
$$
或者说$D$中任两点都可以用$D$中的直线连接起来.设$z = f(x,y) \in C^{n+1} (D)$,$M_0(x_0,y_0) \in D$,$M(x,y) = M(x_0 + h, y_0 + k) \in D$,点$Q(x_1,y_1)$在$M_0$和$M$之间,即$Q$在$M_0$和$M$的连线上.

由$\overrightarrow{M_0Q} \parallel \overrightarrow{M_0M}$,即
$$(x_1 - x_0,y_1 - y_0) \parallel (h,k) \Leftrightarrow \frac{x_1 - x_0}{h} = \frac{y_1 - y_0}{k} := t \Leftrightarrow \begin{cases}
    x_1 = x_0 + th,\\
    y_1 = y_0 + tk.
\end{cases}, t \in [0,1]$$
得
$$
f(Q) = f(x_1,y_1) = f(x_0 + th, y_0 + tk) := \varphi(t) \in C^{n+1}[0,1]
$$  
利用$\varphi(t)$在$t=0$处的$n$阶Taylor公式
$$
\varphi(t) = \sum_{m=0}^n \frac{\varphi^{(m)}(0)}{m!}t^m + \frac{\varphi^{(n+1)}(\theta t)}{(n+1)!}t^{n+1}, \quad t \in [0,1], \theta \in (0,1)
$$
取$t=1$,得到
$$
f(x_0 + h, y_0 + k) = f(x,y) = \varphi(1) = \sum_{m=0}^n \frac{\varphi^{(m)}(0)}{m!} + \frac{\varphi^{(n+1)}(\theta)}{(n+1)!}
$$
其中
\begin{align*}
    \varphi(0) &= f(M_0)\\
    \varphi'(0) &= \left( f_x'(x_0+ ht, y_0 + kt)h + f_y'(x_0 + th, y_0 + kt)k \right)_{t=0}\\
    &= \left( h \parfrac{f}{x} + k \parfrac{f}{y} \right)_{M_0}:= \left(h \parfrac{}{x} + k \parfrac{}{y} \right)f(M_0)\\
    \varphi''(0) &= \left( \parfrac[2]{f(x_0+ht, y_0+kt)}{x}h^2 + 2 \parfrac{f}{x,y}hk + \parfrac[2]{f(x_0+ht, y_0+kt)}{y}k^2 \right)_{t=0}\\
    &= \left( h^2 \parfrac[2]{}{x^2} + 2\parfrac[2]{}{x,y}hk + k^2 \parfrac[2]{}{y^2} \right)f(M_0)\\
    :&= \left( h \parfrac{}{x} + k \parfrac{}{y} \right)^2 f(M_0)
\end{align*}
以此类推,得到
$$
\varphi^{(m)}(0) = \left( h \parfrac{}{x} + k \parfrac{}{y} \right)^m f(M_0), m = 0,1,2,\cdots,n
$$
代入Taylor公式,得到
$$
f(x,y) = f(x_0+h, y_0+k) = \varphi(1) = \sum_{m=0}^n \frac{1}{m!} \left( h \parfrac{}{x} + k \parfrac{}{y} \right)^m f(M_0) + R_n
$$
其中
$$
R_n = \frac{1}{(n+1)!} \left( h \parfrac{}{x} + k \parfrac{}{y} \right)^{n+1} f(x_0 + \theta h, y_0 + \theta k)
$$
其中$\theta \in (0,1)$,$\begin{cases}
    h = x - x_0,\\
    k = y - y_0.
\end{cases}$.

这即为二元函数$f(x,y)$在点$M_0(x_0,y_0)$处的$n$阶Taylor公式.

\begin{example}
    可微函数$z = f(x,y)$在点$M_0(x_0,y_0)$的0阶Taylor公式为
    \begin{align*}
        f(x,y) &= f(x_0,y_0) + f_y'(x_0 + \theta h,y_0 + \theta k) k + f_x'(x_0 + \theta h,y_0 + \theta k)h\\
        &= f(x_0,y_0) + f_x'\left(x_0 + \theta(x-x_0),y_0 + \theta(y-y_0)\right)(x-x_0) \\
        &+ f_y'\left(x_0 + \theta(x-x_0),y_0 + \theta(y-y_0)\right)(y-y_0)
    \end{align*}
    也就是我们可以得到二元函数$z =f(x,y) $的微分中值定理
    \begin{align*}
        f(x,y)-f(x_0,y_0) &= f(x_0,y_0) + f_x'\left(x_0 + \theta(x-x_0),y_0 + \theta(y-y_0)\right)(x-x_0)\\&+ f_y'\left(x_0 + \theta(x-x_0),y_0 + \theta(y-y_0)\right)(y-y_0)
    \end{align*}

\end{example}

\begin{example}
设$z =f(x,y) \in C^3(D)$,则$f(x,y)$在点$M_0(x_0,y_0)$的二阶Taylor公式为
\begin{align*}
    f(x,y) &= f(x_0,y_0) + f_x'(x_0,y_0)(x-x_0) + f_y'(x_0,y_0)(y-y_0)\\
    &+ \frac{1}{2} \left( f_{xx}''(x_0,y_0)(x-x_0)^2 + 2f_{xy}''(x_0,y_0)(x-x_0)(y-y_0) + f_{yy}''(x_0,y_0)(y-y_0)^2 \right) + o(\rho^2)    
\end{align*}
其中$\rho = \sqrt{(x-x_0)^2 + (y-y_0)^2} = \sqrt{h^2 + k^2}$.

设$f_x'(M_0) = 0 = f_y'(M_0)$,$_{xx}''(M_0) = A, f_{xy}''(M_0) = B, f_{yy}''(M_0) = C$,则
\begin{align*}
    f(x,y) - f(x_0,y_0) &= \frac12 \left( A(x-x_0)^2 + 2B(x-x_0)(y-y_0) + C(y-y_0)^2 \right) + o(\rho^2)\\
    &= \frac{A}{2} \left[ \left(x-x_0 + \frac{B}{A}(y-y_0) \right)^2 + \frac{AC-B^2}{A^2}(y-y_0)^2 \right] + o(\rho^2)
\end{align*}
由此进行分析
\begin{enumerate}
    \item 当$\begin{cases}
        A > 0,\\
        AC - B^2 > 0
    \end{cases}$时,$f(x,y) - f(x_0,y_0) \ges 0$恒成立,此时$f(x,y)$在$M_0$处取得极小值.
    \item 当$\begin{cases}
        A < 0,\\
        AC - B^2 > 0
    \end{cases}$时,$f(x,y) - f(x_0,y_0) \les 0$恒成立,此时$f(x,y)$在$M_0$处取得极大值.
    \item 当$\begin{cases}
        AC - B^2 < 0
    \end{cases}$时,$f(x_0,y_0)$不是$f$的极值.
\end{enumerate}
\end{example}

\begin{theorem}
    设$z = f(x,y) \in C^2(D)$,$D$是凸区域,$M_0(x_0,y_0) \in D$且$M_0$是$f$的一个驻点,即
    $$f_x'(M_0) = 0 = f_y'(M_0)$$

    令$1+ f(M_0) = \begin{pmatrix}
        A & B\\
        B & C
    \end{pmatrix}$称之为$f$在驻点$M_0$处的Hessian矩阵,则
    \begin{enumerate}
        \item 当$Hf(M_0) > 0$,即矩阵正定,一切顺序主子式均大于0,则$f$在$M_0$处取得极小值.即$\begin{cases}
            A > 0,\\
            AC - B^2 > 0
        \end{cases}$时,$f$在$M_0$处取得极小值.
        \item 当$Hf(M_0) < 0$,即矩阵负定,一切顺序主子式交替正负,则$f$在$M_0$处取得极大值.即$\begin{cases}
            A < 0,\\
            AC - B^2 > 0
        \end{cases}$时,$f$在$M_0$处取得极大值.
        \item $\Delta = AC-B^2<0$时,$f$在$M_0$处不取得极值.
        \item 当$Hf(M_0) = 0$,即矩阵不定,则$f$在$M_0$处是否取得极值不确定.要使用更高阶的Taylor公式进行讨论.
    \end{enumerate}
\end{theorem}

\begin{example}
    \begin{enumerate}
    \item 设$f(x,y) = x^4+ y^4, f(0,0) = 0$.
    $$ f_x'(0,0) = 4x^3\bigg|_{(0,0)} = 0, quad f_y'(0,0) = 4y^3\bigg|_{(0,0)} = 0$$
    $$ f_{xx}''(0,0) = 12x^2\bigg|_{(0,0)} = 0, quad f_{xy}''(0,0) = 0, quad f_{yy}''(0,0) = 12y^2\bigg|_{(0,0)} = 0$$
    则$\Delta = AC - B^2 = 0$,但是不难看出$f(x,y)$在$(0,0)$处取得极小值.
    \item 设$f(x,y) = x^3 + y^3, f(0,0) = 0$.
    $$ f_x'(0,0) = 3x^2\bigg|_{(0,0)} = 0, quad f_y'(0,0) = 3y^2\bigg|_{(0,0)} = 0$$
    $$ f_{xx}''(0,0) = 6x\bigg|_{(0,0)} = 0, quad f_{xy}''(0,0) = 0, quad f_{yy}''(0,0) = 6y\bigg|_{(0,0)} = 0$$
    则$\Delta = AC - B^2 = 0$,但是不难看出$f(x,y)$在$(0,0)$处不取得极值.
    \end{enumerate}
\end{example}

\section{例题}

\begin{example}
    将$f(x,y) = \frac{1}{1-x-y+xy}$在点$M_0(0,0)$处分别展成零阶,一阶,二阶Taylor公式.
\end{example}

\begin{solution}
\begin{enumerate}
    \item 零阶Taylor公式
    \begin{align*}
        f(x,y) &= f(0,0) + (x-0) f_x'(0 + \theta(x-0),0 + \theta(y-0)) + (y-0) f_y'(0 + \theta(x-0),0 + \theta(y-0))\\
        &1+xf_x'(\theta x, \theta y) + yf_y'(\theta x, \theta y), \theta \in (0,1)
    \end{align*}
    而$f(x,y) = \frac{1}{(1-x)(1-y)}$,则
    $$f_x'(\theta x, \theta y) = \frac{1}{(1-\theta x)^2(1-\theta y)}, quad f_y'(\theta x, \theta y) = \frac{1}{(1-\theta x)(1-\theta y)^2}$$
    代入得零阶Taylor公式为
    $$f(x,y) = 1 + \frac{x}{(1-\theta x)^2}\frac{1}{1-\theta y} + \frac{y}{1-\theta x}\frac{1}{(1-\theta y)^2}$$
    \item 高阶Taylor公式
    
    利用$$\frac{1}{1-x} = 1 + x + o(x) = 1 + x + x^2 + o(x^2) = 1 + x + x^2 + x^3 + o(x^3)$$
    $$\frac{1}{1-y} = 1 + y + o(y) = 1 + y + y^2 + o(y^2) = 1 + y + y^2 + y^3 + o(y^3)$$
    得到$f(x,y)$在点$O(0,0)$处的一阶,二阶,三阶Taylor公式为
    \begin{align*}
        f(x,y)&= (1+x+o(x))(1+y+o(y)) = 1 + x +y +R_1 = 1 + \frac{x^2-y^2}{x-y} + R_1\\
        f(x,y)&= (1+x+x^2+o(x^2))(1+y+y^2+o(y^2)) \\ &= 1 + x + y + x^2 + xy + y^2 + R_2 = 1+\frac{x^2-y^2}{x-y} + \frac{x^3-y^3}{x-y} + R_2\\
        f(x,y)&= (1+x+x^2+x^3+o(x^3))(1+y+y^2+y^3+o(y^3))\\
        &= 1 + x + y + x^2 + xy + y^2 + x^3 + x^2y + xy^2 + y^3 + R_3\\
        &= 1 + \frac{x^2-y^2}{x-y} + \frac{x^3-y^3}{x-y} + \frac{x^4-y^4}{x-y} + R_3
    \end{align*}
    以此类推可以得到$f(x,y)$在点$O(0,0)$处的$n$阶Taylor公式为
    $$f(x,y) = 1 + (x+y) + \frac{x^3-y^3}{x-y} + \frac{x^4-y^4}{x-y} + \cdots + \frac{x^n-y^n}{x-y} + R_n$$
\end{enumerate}
\end{solution}



\begin{example}
    求$f(x,y) = x^3-y^3+3x^2+3y^2-9x$的所有极值.
\end{example}

\begin{solution}
    先求$f(x,y)$的驻点,即求$f_x'(x,y) = 0 = f_y'(x,y)$,得到
    $$\begin{cases}
        f_x'(x,y) = 3x^2 + 6x - 9 = 0,\\
        f_y'(x,y) = -3y^2 + 6y = 0.
    \end{cases}$$
    解得$f(x,y)$的驻点为$M_1(1,0),M_2(1,2),M_3(-3,0),M_4(-3,2)$.
    对每个驻点,计算$f(x,y)$的Hessian矩阵,根据Hessian矩阵的正定性,负定性,不定性来判断极值.

    由$$f_{xx}''=6x+6, f_{xy}''=0, f_{yy}''=-6y+6$$
    \begin{enumerate}
        \item 在$M_1(1,0)$处,
        $$A = 12, \quad B = 0, \quad C = 6$$
        由此得到$\Delta = AC - B^2 = 72 > 0,A>0$,故$f(M_1) = f(1,0) = -5$为极小值.
        \item 在$M_2(1,2)$处,
        $$A = 12, \quad B = 0, \quad C = -6$$
        由此得到$\Delta = AC - B^2 = -72 < 0$,故$f(M_2) = f(1,2) = 5$不是极值.
        \item 在$M_3(-3,0)$处,
        $$A = -12, \quad B = 0, \quad C = 6$$
        由此得到$\Delta = AC - B^2 = -72 < 0$,故$f(M_3) = f(-3,0) = 45$不是极值.
        \item 在$M_4(-3,2)$处,
        $$A = -12, \quad B = 0, \quad C = -6$$
        由此得到$\Delta = AC - B^2 = 72 > 0,A<0$,故$f(M_4) = f(-3,2) = 31$为极大值.
    \end{enumerate}
\end{solution}










\section{对称矩阵$A$的正定性与负定性}

\begin{definition}
    [对称矩阵]

    设$A = \begin{pmatrix}
    a_{11} & a_{12} & a_{13}\\
    a_{21} & a_{22} & a_{23}\\
    a_{31} & a_{32} & a_{33}
\end{pmatrix}$且$a_{12} = a_{21}, a_{13} = a_{31}, a_{23} = a_{32}$,则$A$称之为对称矩阵,此时$A = A^T$.
\end{definition}

\begin{definition}
    [正定矩阵与负定矩阵]
    $A$为$n$阶对称矩阵,则
    \begin{enumerate}
        \item 若$A$的顺序主子式全大于零,即$|a_{11}| = a_{11} > 0, \begin{vmatrix}
            a_{11} & a_{12}\\
            a_{21} & a_{22}
        \end{vmatrix} >0,\begin{vmatrix}
            a_{11} & a_{12} & a_{13}\\
            a_{21} & a_{22} & a_{23}\\
            a_{31} & a_{32} & a_{33}
        \end{vmatrix} = |A| > 0 $,则称$A$为正定矩阵.
        \item 若$A$的奇数阶顺序主子式全小于零,偶数阶顺序主子式全大于零,即$|a_{11}| = a_{11} < 0, \begin{vmatrix}
            a_{11} & a_{12}\\
            a_{21} & a_{22}
        \end{vmatrix} <0,\begin{vmatrix}
            a_{11} & a_{12} & a_{13}\\
            a_{21} & a_{22} & a_{23}\\
            a_{31} & a_{32} & a_{33}
        \end{vmatrix} = |A| > 0 $,则称$A$为负定矩阵.
    \end{enumerate}

    更高阶的对称矩阵的正定性与负定性的定义类似.
\end{definition}

\begin{homework}
    ex9.5:2(2),3,4(1)(3)(7),7(1)(3)(4).
\end{homework}












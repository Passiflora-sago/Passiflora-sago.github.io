\setcounter{chapter}{13} % 设置章节计数器
\chapter{多元函数的极值与最值}

\section{多元函数极值必要条件与充分条件}

\begin{theorem}
    [可微函数有极值的必要条件]

    设$z=f(x,y)$在$\overline U (M_0,\delta)$中可微,且$f(M_0)$为$f$的极值,则$f_x'(M_0) = f_y'(M_0) = 0$.
\end{theorem}

\begin{proof}
    设$M_0(x_0,y_0)$为$f$的极小值点.考虑函数
    $$
    g(x) = f(x,y_0), x \in \overline U {(x_0-\delta,x_0+\delta)}
    $$
    则$g(x)$在$x_0$处取得极小值,故$g'(x_0) = 0$,即$f_x'(M_0) = 0$.同理可证$f_y'(M_0) = 0$.
\end{proof}

这表示可微函数的极值点必是驻点,可推广到$n$元函数.

\begin{theorem}
    [二阶连续可微函数有极值的充分条件]

    设$z=f(x,y) \in C^2( \overline U (M_0,\delta))$,且$M_0(x_0,y_0)$为$f$的驻点,即$f_x'(M_0) = f_y'(M_0) = 0$,记$A = f_{xx}''(M_0),B = f_{xy}''(M_0),C = f_{yy}''(M_0)$,$Hf(M_0) = \begin{pmatrix}
        A & B\\
        B & C
    \end{pmatrix}$,则
    \begin{enumerate}
        \item $Hf(M_0) > 0$,Hessian矩阵正定,即$A > 0,AC - B^2 > 0$,则$f$在$M_0$处取得极小值.
        \item $Hf(M_0) < 0$,Hessian矩阵负定,即$A < 0,AC - B^2 > 0$,则$f$在$M_0$处取得极大值.
        \item $AC - B^2 < 0$,Hessian矩阵不定,则$f$在$M_0$处不取得极值.
    \end{enumerate}
\end{theorem}

\begin{proof}
    $\forall (x,y) \in \overline U(M_0,\delta)$,记$\begin{cases}
        h = x - x_0 = \Delta x,\\
        k = y - y_0 = \Delta y.
    \end{cases}$,由二阶Taylor公式
    $$
    f(x,y)-f(x_0,y_0) = \frac{1}{2} \left( Ah^2 + 2Bhk + Ck^2 \right) + o(\rho^2)
    $$
    其中$\rho = \sqrt{h^2 + k^2}$.

    设 $h, k$ 不全为零,引进变量
\[
u = \frac{h}{\sqrt{h^2 + k^2}} = \frac{h}{\rho}, \quad v = \frac{k}{\sqrt{h^2 + k^2}} = \frac{k}{\rho}
\]
则
\[
Q(h, k) = \rho^2 \varphi(u, v) = \rho^2 (Au^2 + 2Buv + Cv^2)
\]
其中 $\varphi(u, v) = Au^2 + 2Buv + Cv^2$ 是定义在单位圆周 $u^2 + v^2 = 1$ 上的连续函数.  
当 $Q$ 正定时,$\varphi(u, v) > 0$ 对圆周上任意点 $(u, v)$ 成立.而单位圆周是一个有界闭集,所以 $\varphi(u, v)$ 在圆周上的最小值 $m$ 也是正的,且 $Q \ges m\rho^2$,即
\[
f(x, y) - f(x_0, y_0) = \frac{1}{2}(Ah^2 + 2Bhk + Ck^2) + o(\rho^2) \ges \rho^2 \left( \frac{1}{2} m + \frac{o(\rho^2)}{\rho^2} \right)
\]

注意到上式右端括号内的值当 $\rho$ 充分小时,一定是正的,所以有 $f(x, y) - f(x_0, y_0) > 0$.  
即 $(x_0, y_0)$ 是极小值点.对于负定的情形,证明是完全类似的.

当$\Delta < 0 $时,因为 $Q$ 是不定的,所以存在点 $(h_1, k_1), (h_2, k_2)$ 使得
\[
Q(h_1, k_1) = \rho_1^2 \varphi(u_1, v_1) > 0, \quad Q(h_2, k_2) = \rho_2^2 \varphi(u_2, v_2) < 0
\]

上述条件等价于 $\varphi(u_1, v_1) > 0, \quad \varphi(u_2, v_2) < 0$,所以存在一个数 $m$ 使得
\[
\varphi(u_1, v_1) > m > 0, \quad \varphi(u_2, v_2) < -m < 0
\]
即
\[
Q(h_1, k_1) > m\rho_1^2, \quad Q(h_2, k_2) < -m\rho_2^2
\]

令 $h = th_1, \, k = tk_1, \, 0 \les t \les 1$,则 $\rho = t\rho_1$,$Q(h, k) = t^2 Q(h_1, k_1) > mt^2 \rho_1^2$,  
\[
f(x_0 + th_1, y_0 + tk_1) - f(x_0, y_0) = \frac{1}{2} Q(th_1, tk_1) + o(t^2 \rho_1^2) > t^2 \left( \frac{1}{2} m + \frac{o(t^2)}{t^2 \rho_1^2} \right)
\]

所以对充分小的 $t$,上式右端括号内大于零.也就是说 $f(x_0 + th_1, y_0 + tk_1) - f(x_0, y_0) > 0$.  
同样的道理,存在一个充分小的 $t$,使得 $f(x_0 + th_2, y_0 + tk_2) - f(x_0, y_0) < 0$.

综上分析,在 $(x_0, y_0)$ 的任意小的邻域内,既有大于 $f(x_0, y_0)$ 的值,又有小于 $f(x_0, y_0)$ 的值,所以 $(x_0, y_0)$ 不是极值点.

关于半正定情形无法判断,在此不再讨论.不过我们可以用几个例子来说明半正定情形无法判别极值.

\end{proof}

\begin{example}
    考虑下列函数在$O(0,0)$处取值的情况:
    \begin{enumerate}
        \item $z=f(x,y) = x^2 + y^2$,
        \item $z=f(x,y) = x^3 + y^3$,
        \item $z=f(x,y) = x^4 + y^4$.
        \item $z=f(x,y) = \sqrt{x^2 + y^2}$.
        \item $z=f(x,y) = xy$.
    \end{enumerate}
\end{example}

\begin{solution}
    \begin{enumerate}
        \item $O(0,0)$为$f$的驻点,且
        $$ A = f_{xx}''(0,0) = 2, B = f_{xy}''(0,0) = 0, C = f_{yy}''(0,0) = 2$$
        则
        $$Hf(0,0) = \begin{pmatrix}
            2 & 0\\
            0 & 2
        \end{pmatrix} > 0$$
        故$f(0,0) = 0$为极小值.
        \item $$\begin{cases}
            f_x' = 3x^2 = 0,\\
            f_y' = 3y^2 = 0.
        \end{cases}$$.故$O(0,0)$为$f$的驻点,且
        $$ A = f_{xx}''(0,0) = 0, B = f_{xy}''(0,0) = 0, C = f_{yy}''(0,0) = 0$$
        则
        $\Delta = AC - B^2 = 0$,无法由此判断$f(0,0)$是否为极值.

        对于任意的$\delta > 0 $,在$\overline U(0,\delta)$之中,总有$M_1,M_2 \in U(0,\delta)$,使得
        $$f(M_1) > 0 = f(0,0) = f(M_2)$$
        故$O(0,0)$不是极值.
        \item $O(0,0)$为$f$的驻点,且
        $$ A = f_{xx}''(0,0) = 0, B = f_{xy}''(0,0) = 0, C = f_{yy}''(0,0) = 0$$
        无法由此判断$f(0,0)$是否为极值.

        但是不难看出$\forall (x,y) , f(x,y) \ges 0 = f(0,0)$,故$O(0,0)$为最小值,当然是极小值.
        \item $f(0,0) = 0$为$f$的极小值,也是$f$的最小值,但是
        $$f_x'(0,0) = \lim_{h \to 0} \frac{f(h,0) - f(0,0)}{h} = \lim_{h \to 0} \frac{\sqrt{h^2}}{h} $$不存在,故$O(0,0)$不是驻点.
        \item 从$\begin{cases}
            f_x' = y = 0,\\
            f_y' = x = 0.
        \end{cases}$ 得到驻点$O(0,0)$,但
        $\forall \delta > 0 , \exists (\frac{\delta}{2},\frac{\delta}{2}), (\frac{\delta}{2},-\frac{\delta}{2}) \in U(0,\delta)$
        使得$f(\frac{\delta}{2},\frac{\delta}{2}) = \frac{\delta^2}{4} > 0 = f(0,0) = f(\frac{\delta}{2},-\frac{\delta}{2})$,故$O(0,0)$不是极值.
    \end{enumerate}
\end{solution}

\begin{remark}
    由上述例子可知,
    \begin{itemize}
        \item 驻点可以是极值点,也可以不是极值点.
        \item 极值点可以是驻点,也可以不是驻点.
        \item $\Delta = AC - B^2 = 0 $时,可以有极值,也可以没有极值.
    \end{itemize}
\end{remark}

\section{$n$元函数的极值与最值}

设$u=f(x_1,x_2,\cdots,x_n)$,其中$x_1,x_2,\cdots,x_n$是$n$个自变量,$f$是一个$n$元函数.
$u=f(x_1,x_2,\cdots,x_n) \in C^3(D)$,其中$D$是凸区域.$M_0(x_{10},x_{20},\cdots,x_{n0}) \in D$且$M_0$是$f$的一个驻点,即
$$
f_{x_1}'(M_0) = f_{x_2}'(M_0) = \cdots = f_{x_n}'(M_0) = 0
$$

记$a_{ij} = \parfrac{f}{x_i,x_j} \bigg|_{M_0} = a_{ji}, i,j = 1,2,\cdots,n$,记$$Hf(M_0) := \begin{vmatrix}
    a_{11} & a_{12} & \cdots & a_{1n}\\
    a_{21} & a_{22} & \cdots & a_{2n}\\
    \vdots & \vdots & \ddots & \vdots\\
    a_{n1} & a_{n2} & \cdots & a_{nn}
\end{vmatrix}$$则
\begin{enumerate}
    \item $Hf(M_0) > 0 $,即$Hf(M_0)$正定时,$f(M_0)$为$f$的极小值.
    \item $Hf(M_0) < 0 $,即$Hf(M_0)$负定时,$f(M_0)$为$f$的极大值.
    \item 若$f$在凸的开区域$D$中仅有一个极小(大)值,且无极大(小)值.则此极值为$f$的最小(大)值.
    \item 若$D$是有界的闭区域.则$f$在$D$中科同时取最大值与最小值.可先求出$f$在$D$内的所有可能的驻点$M_1,M_2,\cdots,M_k$,再求出$f$在这些驻点上的函数值$f(M_1),f(M_2),\cdots,f(M_k)$,其中$f(M_1),f(M_2),\cdots,f(M_k)$中的最大值与最小值即为$f$在$D$中的最大值与最小值.
\end{enumerate}

\section{例题}

\begin{example}
    在椭球面$\Sigma: \frac{x^2}{a^2} + \frac{y^2}{b^2} + \frac{z^2}{c^2} = 1$上取一点$M_0(x_0,y_0,z_0)$,求$M_0$处的切平面方程与三个坐标面围成的四面体$\Omega$的体积的最小值.
\end{example}

\begin{solution}
    设$M_0$位于第一卦限,则$\pi$为$$\frac{x_0x}{a^2} + \frac{y_0y}{b^2} + \frac{z_0z}{c^2} = 1$$
    则$V(\Omega) = \frac16 \left( \frac{a^2}{x_0} \right) \left( \frac{b^2}{y_0} \right) \left( \frac{c^2}{z_0} \right) = \frac{abc}{6} \frac{1}{\sqrt{\left(\frac{x_0}{a}\right)^2 + \left(\frac{y_0}{b}\right)^2 + \left(\frac{z_0}{c}\right)^2}}$
    再利用平均值不等式,得到
    $$\frac{abc}{6} \frac{1}{\sqrt{\left(\frac{x_0}{a}\right)^2 + \left(\frac{y_0}{b}\right)^2 + \left(\frac{z_0}{c}\right)^2}} \ges \frac{abc}{6} \frac{1}{\sqrt{3}}$$
    故$V(\Omega)$的最小值为$\frac{abc}{6\sqrt{3}}$.此时切平面方程为$\frac{x}{a} + \frac{y}{b} + \frac{z}{c} = \sqrt{3}$.
\end{solution}

\begin{example}
    证明$z=f(x,y) = \left( x^2 + y^2 \right) \e^{-(x+y)}$在$D :\begin{cases}
        x \ges 0,\\
        y \ges 0,
    \end{cases}$中的最大值为$4\e^{-2}$.
\end{example}

\begin{solution}
    \begin{enumerate}
        \item 在$D$内部仅有疑点$M_1(1,1)$,
        \item 在边界$y=0$上$f(x,0) = x^2\e^{-x}$有疑点$x_1 = 0,x_2 = 2$,
        \item 在边界$x=0$上$f(0,y) = y^2\e^{-y}$有疑点$y_1 = 0,y_2 = 2$.
    \end{enumerate}
    且
    $$f(0,0) = 0,\quad f(2,0) = 4 \e^{-2} = f(0,2), \quad f(1,1) = 2\e^{-2}$$
    且$\lim_{\substack{x \to \infty \\ y \to \infty}} \left( x^2 + y^2 \right) \e^{-(x+y)} = 0$,故$f(x,y)$在$D$中的最大值为$4\e^{-2}$.
\end{solution}

\begin{homework}
    ex9.5:7(2)(5),8,11(2)(4),17;CH9:6,14.
\end{homework}











\setcounter{chapter}{9} % 设置章节计数器

\chapter{多元函数微分法习题课(1)}

\section{例题}

\begin{example}
    设方程:$u^3 - 3(x+y) u^2 + z^3 = 0$确定了隐函数$u = f(x,y,z)$,求$\dif u$.
\end{example}

\begin{solution}
解法一:

原方程两边取全微分$\dif$,得
\begin{align*}
    & \diff (u^3 - 3(x+y) u^2 + z^3) = \diff (0) = 0 \\
    \Rightarrow &\diff(u^3) - 3 \diff((x+y)u^2) + \diff(z^3) = 0 \\
    \Rightarrow & 3u^2 \dif u - 3u^2 (\dif x + \dif y) - 3(x+y)2u \dif u + 3z^2 \dif z = 0 \\
\end{align*}

整理得
$$
\dif u = \frac{3u^2(\dif x + \dif y) + 3z^2 \dif z}{3u^2 + 6(x+y)u} = \frac{u^2 \dif x + u^2 \dif y - z^2 \dif z}{u^2 - 2(x+y)u}
$$
\end{solution}


\begin{solution}
解法二:

令$F(x,y,z,u) = u^3 - 3(x+y)u^2 + z^3 $,其中$x,y,z,u$地位相同,则$F_u' = 3u^2 - 6(x+y)u$,$F_x' = -3u^2$,$F_y' = -3u^2$,$F_z' = 3z^2$.从而
\begin{align*}
    \parfrac{u}{x} &= - \frac{F_x'}{F_u'} = \frac{u^2}{-2(x+y)u + u^2}; \\
    \parfrac{u}{y} &= - \frac{F_y'}{F_u'} = \frac{u^2}{-2(x+y)u + u^2}; \\
    \parfrac{u}{z} &= \frac{F_z'}{F_u'} = \frac{z^2}{-2(x+y)u + u^2}.
\end{align*}
因此
$$
\dif u = \parfrac{u}{x} \dif x + \parfrac{u}{y} \dif y + \parfrac{u}{z} \dif z = \frac{u^2 \dif x + u^2 \dif y - z^2 \dif z}{u^2 - 2(x+y)u}.
$$
\end{solution}

\begin{solution}
解法三:

原方程两边分别对$x,y,z$求偏导数,解出$\frac{\partial u}{\partial x},\frac{\partial u}{\partial y},\frac{\partial u}{\partial z}$再代入$\dif u = \frac{\partial u}{\partial x} \dif x + \frac{\partial u}{\partial y} \dif y + \frac{\partial u}{\partial z} \dif z$即可.
\end{solution}

\begin{example}
    试证明方程
    $$
    \parfrac{^2 u}{x^2} + 2 \parfrac{^2 u}{x,y} - 3 \parfrac{^2 u}{y^2} + 2 \parfrac{u}{x} + 6 \parfrac{u}{y} = 0
    $$
    在线性变换
    $$
    \begin{cases}
        \xi = x + y, \\
        \eta = 3x - y
    \end{cases}
    $$
    下可以化简为
    $$
    \parfrac{^2 u}{\xi,\eta} + \frac{1}{2} \parfrac{u}{\xi} = 0
    $$
\end{example}

\begin{solution}
    证法一:

    从线性变换$\begin{cases}
        \xi = x + y, \\
        \eta = 3x - y
    \end{cases}$可得$x = \frac{1}{4}(\xi + \eta)$,$y = \frac{1}{4}(3\xi - \eta)$,因此$\parfrac{x}{\xi} = \frac14,\parfrac{x}{\eta} = \frac14,\parfrac{y}{\xi} = \frac34,\parfrac{y}{\eta} = -\frac14$.由此得
    \begin{align*}
        \parfrac{u}{\xi} &= \parfrac{u}{x} \parfrac{x}{\xi} + \parfrac{u}{y} \parfrac{y}{\xi} = \frac14 \parfrac{u}{x} + \frac34 \parfrac{u}{y} \\
        \Rightarrow \parfrac{^2 u}{\eta,\xi} &= \left( \parfrac{u}{\xi} \right)'_{\eta} = \left( \frac14 \parfrac{u}{x} \right)_\eta' + \left( \frac34 \parfrac{u}{y} \right)_\eta'\\
        &= \frac14 \left( \parfrac{^2 u}{x^2} \parfrac{x}{\eta} + \parfrac{^2 u}{x,y} \parfrac{y}{\eta} \right) + \frac34 \left( \parfrac{^2 u}{x,y} \parfrac{x}{\eta} + \parfrac{^2 u}{y^2} \parfrac{y}{\eta} \right)\\
        &= \frac14 \left( \parfrac{^2 u}{x^2} \frac14 + \parfrac{^2 u}{x,y} \left( -\frac14 \right) \right) + \frac34 \left( \parfrac{^2 u}{x,y} \frac14 + \parfrac{^2 u}{y^2} \left( -\frac14 \right) \right)\\
        &= \frac{1}{16} \left( \parfrac{^2 u}{x^2} + 2 \parfrac{^2 u}{x,y} - 3 \parfrac{^2 u}{y^2} \right)
    \end{align*}

    即有$$
    \begin{cases}
        &\parfrac{^2 u}{x^2} + 2 \parfrac{^2 u}{x,y} - 3 \parfrac{^2 u}{y^2} = 16 \parfrac{^2 u}{\eta,\xi} \\
        & \parfrac{u}{x} + 3 \parfrac{u}{y} = 4 \parfrac{u}{\xi}
    \end{cases}$$

    故原偏微分方程化简为
    \begin{align*}
        &16 \parfrac{^2 u}{\eta,\xi} + 2 \left( 4 \parfrac{u}{\xi} \right) = 0 \\
        \Rightarrow &\parfrac{^2 u}{\eta,\xi} + \frac12 \parfrac{u}{\xi} = 0
    \end{align*}
\end{solution}

\begin{solution}
    解法二:

    从线性变换$\begin{cases}
        \xi = x + y, \\
        \eta = 3x - y
    \end{cases} \Rightarrow \parfrac{\xi}{x} = 1, \parfrac{\xi}{y} = 1, \parfrac{\eta}{x} = 3, \parfrac{\eta}{y} = -1$.而$u(x,y)$通过中间变量可视为$\xi,\eta$的函数,从而
    $$
    \begin{cases}
        \parfrac{u}{x} = \parfrac{u}{\xi} \parfrac{\xi}{x} + \parfrac{u}{\eta} \parfrac{\eta}{x} = \parfrac{u}{\xi} + 3 \parfrac{u}{\eta}, \\
        \parfrac{u}{y} = \parfrac{u}{\xi} \parfrac{\xi}{y} + \parfrac{u}{\eta} \parfrac{\eta}{y} = \parfrac{u}{\xi} - \parfrac{u}{\eta}.
    \end{cases},
    $$
    从而
    $$
    \begin{cases}
        \parfrac{^2 u}{x^2} = \left( \parfrac{u}{\xi} \right)_x' + 3 \left( \parfrac{u}{\eta} \right)_x' = \left( \parfrac{^2 u}{\xi^2} + 3 \parfrac{^2 u}{\eta,\xi}  \right) + 3 \left( \parfrac{^2 u}{\xi,\eta} +3 \parfrac{^2 u}{\eta^2} \right), \\
        \parfrac{^2 u}{x,y} = \left( \parfrac{u}{\xi} - \parfrac{u}{\eta} \right)_x' = \left( \parfrac{^2 u}{\xi^2} + 3 \parfrac{^2 u}{\xi,\eta} \right) - \left( \parfrac{^2 u}{\xi,\eta} +3 \parfrac{^2 u}{\eta^2} \right), \\
        \parfrac{^2 u}{y^2} = \left( \parfrac{u}{\xi} - \parfrac{u}{\eta} \right)_y' = \left( \parfrac{^2 u}{\xi^2} - \parfrac{^2 u}{\xi,\eta} \right) - \left( \parfrac{^2 u}{\xi,\eta} - \parfrac{^2 u}{\eta^2} \right).
    \end{cases}
    $$
    即
    $$
    \begin{cases}
        \parfrac{^2 u}{x^2} = \parfrac{^2 u}{\xi^2} + 6 \parfrac{^2 u}{\xi,\eta} + 9 \parfrac{^2 u}{\eta^2}, \\
        \parfrac{^2 u}{x,y} = \parfrac{^2 u}{\xi^2} + 2 \parfrac{^2 u}{\xi,\eta} - 3 \parfrac{^2 u}{\eta^2}, \\
        \parfrac{^2 u}{y^2} = \parfrac{^2 u}{\xi^2} - 2 \parfrac{^2 u}{\xi,\eta} + \parfrac{^2 u}{\eta^2}.
    \end{cases}
    $$
    且
    $$ 2 \parfrac{u}{x} + 6 \parfrac{u}{y} = 2 \left( \parfrac{u}{\xi} + 3 \parfrac{u}{\eta} \right) + 6 \left( \parfrac{u}{\xi} - \parfrac{u}{\eta} \right) = 8 \parfrac{u}{\xi},
    $$
    从而原方程化为
    $$
    \parfrac{^2 u}{\xi^2} (1+2-3) + \parfrac{^2 u}{\xi,\eta} (6+4+6) + \parfrac{^2 u}{\eta^2} (9-6-3) + 8 \parfrac{u}{\xi} = 0,
    $$
    即
    $$
    \parfrac{^2 u}{\xi,\eta} + \frac12 \parfrac{u}{\xi} = 0.
    $$
\end{solution}

\begin{example}
    设$u = f(x,y,z), \varphi(x^2,\e^y,z)=0, y = \sin x$,且$f,\varphi \in C^1, \parfrac{\varphi}{z} \neq 0$,求$\frac{\dif u}{\dif x}$.
\end{example}

\begin{solution}
    从$\varphi(x^2, \e^{\sin x},z) = 0$及$\varphi_z' \neq 0$可知,由方程$\varphi(x^2, \e^{\sin x},z) = 0$可确定$z $ 是$x,y$的隐函数,从而$z$是$x$的复合函数.故从$u = f(x,y,z)$知,$u$是$x$的一元函数.

    \begin{remark}
        助教注:这个地方可以理解为由隐函数定理$F(x,y,z) = \varphi(x^2, \e^y, z)=0$,确定了隐函数$z = z(x,y)$,从而$u =f(x,y,z) = f(x,y,z(x,y))=f(x,\sin x, z(x,\sin x))$确定了$u$是$x$的函数.
    \end{remark}
    $$
    \frac{\dif u}{\dif x} = f_1' \cdot 1 + f_2' \cdot y_x' + f_3' \cdot z_x' = f_1' + f_2' \cdot \cos x \cdot 2x + f_3' \cdot z_x'.
    $$
    令$F(x,y,z) = \varphi(x^2, \e^y, z)$,则
    $$
    \begin{cases}
        F_x'(x,y,z) = \varphi_1' \cdot 2x + \varphi_2' \e^{\sin x} \cos x,\\
        F_z'(x,y,z) = \varphi_3' \cdot 1 = \varphi_3'.
    \end{cases}
    $$
    故
    $z_x' = - \frac{F_x'}{F_z'} = - \frac{\varphi_1' \cdot 2x + \varphi_2' \e^{\sin x} \cos x}{\varphi_3'}$.代入$\frac{\dif u}{\dif x}$即有
    $$
    \frac{\dif u}{\dif x} = f_1' + f_2' \cdot y_x' + f_3' \cdot \left( - \frac{\varphi_1' \cdot 2x + \varphi_2' \e^{\sin x} \cos x}{\varphi_3'} \right).
    $$

    \begin{remark}
        助教注: 这里老师写的确实很模糊.我们要区分两个式子和他们分别的含义.
        \begin{enumerate}
            \item 令$F(x,y,z) = \varphi(x^2, \e^y, z)$,则$F(x,y,z) = 0$确定了$z = z(x,y)$,其中$z_x' = - \frac{F_x'}{F_z'}$.这时候$z_x'$表示的是$\parfrac{z}{x}$.
            $F_x'(x,y,z) = \varphi_1' \cdot 2x, F_z'(x,y,z) = \varphi_3'$,从而$z_x' = - \frac{F_x'}{F_z'} = - \frac{2x \varphi_1'}{\varphi_3'}$.
            \item 令$F(x,z) = \varphi(x^2, \e^{\sin x}, z)$,则$F(x,z) = 0$确定了$z = z(x)$.这时候$z_x'$表示的是$\frac{\dif z}{\dif x}$.
            $F_x'(x,z) = \varphi_1' \cdot 2x + \varphi_2' \e^{\sin x} \cos x, F_z'(x,z) = \varphi_3'$,从而$z_x' = - \frac{F_x'}{F_z'} = - \frac{2x \varphi_1' + \varphi_2' \e^{\sin x} \cos x}{\varphi_3'}$.
        \end{enumerate}
        老师要表示的实际是第二种情况,即$z = z(x)$.只不过写成的形式看起来像是第一种情况.
    \end{remark}
    \begin{remark}
        正是因为老师这里写模糊了,所以会有疑问为什么当$F(x,z) = \varphi(x^2, \e^{\sin x}, z)$时, $$F_x' = \varphi_1' \cdot 2x + \varphi_2' \e^{\sin x} \cos x$$而不是$$F_x' = \varphi_1' \cdot 2x + \varphi_2' \cdot \e^{\sin x} \cos x+\varphi_3' \cdot z_x'$$
        后者是$F(x) = \varphi(x^2, \e^{\sin x}, z(x))$时的$F_x' = F'(x)$.而为了求$\frac{\dif z}{\dif x}$,我们需要对$F(x,z)$这个函数利用隐函数定理,此时$x,z$都是这个$F(x,z)$的自变量,因此$F_x'=\varphi_1' \cdot 2x + \varphi_2' \e^{\sin x} \cos x$.
    \end{remark}
\end{solution}

\begin{example}
    证明:全微分也具有一阶微分形式不变性,即,若$f(x,y)$可微,则不论$x,y$是自变量还是中间变量,则$z = f(x,y)$,总有
    $$ \dif z= \dif f(x,y) = \parfrac{z}{x} \dif x + \parfrac{z}{y} \dif y = f_x' \dif x + f_y' \dif y. $$
\end{example}

\begin{proof}
    \begin{enumerate}
        \item 当$x,y$是自变量时,显然有$\dif z = f_x' \dif x + f_y' \dif y$.
        \item 当$x,y$是中间变量时,设$\begin{cases}
            x = g(s,t),\\
            y = h(s,t),
        \end{cases}$可微,且$f(g(s,t),h(s,t))$有意义时,$z$通过中间变量$x,y$成为$s,t$的复合函数,且有求偏导数的链式法则如下:
        \begin{align*}
            &\dif x = \parfrac{x}{s} \dif s + \parfrac{x}{t} \dif t,\\
            &\dif y = \parfrac{y}{s} \dif s + \parfrac{y}{t} \dif t.
        \end{align*}
        且
        \begin{align*}
            \dif z&= \parfrac{z}{s} \dif s + \parfrac{z}{t} \dif t = \left( \parfrac{z}{x} \cdot \parfrac{x}{s} + \parfrac{z}{y} \cdot \parfrac{y}{s} \right) \dif s + \left( \parfrac{z}{x} \cdot \parfrac{x}{t} + \parfrac{z}{y} \cdot \parfrac{y}{t} \right) \dif t \\
            &=\parfrac{z}{x} \left(\parfrac{x}{s} \dif s + \parfrac{x}{t} \dif t \right) + \parfrac{z}{y} \left( \parfrac{y}{s} \dif s + \parfrac{y}{t} \dif t \right)\\
            &= \parfrac{z}{x} \dif x+ \parfrac{z}{y} \dif y.
        \end{align*}
        即$x,y$是中间变量时,也有$\dif z = f_x' \dif x + f_y' \dif y$.
    \end{enumerate}
\end{proof}

\begin{remark}
    利用全微分的一阶微分形式不变性,可导出多元可微函数的如下的微分四则运算法则:
    \begin{enumerate}
        \item $\diff ( u \pm v) = \dif u \pm \dif v$;
        \item $\diff (uv) = u \dif v + v \dif u$;
        \item $\diff \left( \frac{u}{v} \right) = \frac{v \dif u - u \dif v}{v^2}$,其中$u,v$均可微,且$v \neq 0$;
    \end{enumerate}
\end{remark}

\begin{proof}
    \begin{enumerate}
        \item 令$f(u,v) = u+v$,则$f(u,v) \in C^1$,从而$f(u,v)$可微,无论$u,v$是自变量还是中间变量,总有
    $$
    \diff( u + v) = \dif f(u,v) = \parfrac{f}{u} \dif u + \parfrac{f}{v} \dif v = \dif u +\dif v
    $$
    从而有$\diff(u \pm v) = \dif u \pm \dif v$.这里$d$是全微分.
    \item 令$f(u,v) = uv$,则$f(u,v) \in C^1$,从而$f(u,v)$可微,无论$u,v$是自变量还是中间变量,总有
    $$
    \diff(uv) = \dif f(u,v) = \parfrac{f}{u} \dif u + \parfrac{f}{v} \dif v = v \dif u + u \dif v
    $$
    从而有$\diff(uv) = u \dif v + v \dif u$.
    \item 令$f(u,v) = \frac{u}{v}$,则$f(u,v) \in C^1$,从而$f(u,v)$可微,无论$u,v$是自变量还是中间变量,总有
    $$
    \diff \left( \frac{u}{v} \right) = \dif f(u,v) = \parfrac{f}{u} \dif u + \parfrac{f}{v} \dif v = \frac{1}{v} \dif u + \left( -\frac{u}{v^2} \right) \dif v = \frac{v \dif u - u \dif v}{v^2}
    $$
    \end{enumerate}
    
\end{proof}

\begin{remark}
    二阶及以上的微分通常没有形式不变性,具体而言,设$f(x,y) \in C^2$,则$z = f(x,y) \Rightarrow \dif z= \parfrac{z}{x} \dif x + \parfrac{z}{y} \dif y$.
    \begin{align*}
        \diff(\dif z) &:= \dif{^2} z = \diff \left( \parfrac{z}{x} \dif x+ \parfrac{z}{y} \dif y \right)\\
        &= \left( \parfrac{z}{x} \dif x+ \parfrac{z}{y} \dif y \right)_x' \dif x + \left( \parfrac{z}{x} \dif x + \parfrac{z}{y} \dif y \right)_y' \dif y\\
        &= \left( \parfrac{^2 z}{x^2} \dif x + \parfrac{^2 z}{x,y} \dif y \right) \dif x + \left( \parfrac{^2 z}{x,y} \dif x + \parfrac{^2 z}{y^2} \dif y \right) \dif y\\
        &= \parfrac{^2 z}{x^2} (\dif x)^2 + 2 \parfrac{^2 z}{x,y} \dif x \dif y + \parfrac{^2 z}{y^2} (\dif y)^2.
    \end{align*}
    $\dif{^2} z$是$x,y$是自变量时的$z = f(x,y)$的二阶微分,而$\dif{^2} z$是$x,y$是中间变量时的$z = f(x,y)$的二阶微分,二者通常不相等.
\end{remark}

\begin{example}
    设$u = u(x,y) , v=v(x,y)$是由方程组
    $$
    \begin{cases}
        u = f(ux,v+y),\\
        v = g(u-x,v^2y)
    \end{cases}
    $$所确定的隐函数组,求变换$\begin{cases}
        u = u(x,y),\\
        v = v(x,y)
    \end{cases}$的Jacobi行列式:
    $$
    \begin{vmatrix}
        u_x' & u_y'\\
        v_x' & v_y'
    \end{vmatrix} := \frac{\partial(u,v)}{\partial(x,y)}  \quad f,g \in C^1.
    $$
\end{example}

\begin{solution}
    令$A = ux, B = v+y,E = u-x, F = v^2y$,则方程组可化为
    $$
    \begin{cases}
        u = f(A,B),\\
        v = g(E,F).
    \end{cases}
    $$
    方程组两边关于$x$求偏导
    $$\begin{cases}
        u_x' = f_1' \cdot ( u + x u_x' ) + f_2' \cdot (v_x' + 0),\\
        v_x' = g_1' \cdot (u_x' - 1) + g_2' \cdot 2v v_x' y.
    \end{cases}$$
    标准化为
    $$
    \begin{cases}
        (x f_1' -1) u_x' + f_2' v_x' = -f_1' u,\\
        g_1' u_x' + (2v g_2' y - 1) v_x' = g_1'.
    \end{cases}
    $$
    令$D = \begin{vmatrix}
        x f_1' - 1 & f_2'\\
        g_1' & 2v g_2' y - 1
    \end{vmatrix}$,则$D \neq 0$,再令
    $$
    D_1 = \begin{vmatrix}
        -f_1' u & f_2'\\
        g_1' & 2v g_2' y - 1
    \end{vmatrix}, \quad D_2 = \begin{vmatrix}
        x f_1' - 1 & -f_1' u\\
        g_1' & g_1'
    \end{vmatrix},
    $$
    由克莱姆法则可得
    $$
    u_x' = \frac{D_1}{D}, \quad v_x' = \frac{D_2}{D}.
    $$
    方程组$\begin{cases}
        u = f(A,B),\\
        v = g(E,F)
    \end{cases}$两边同时对$y$求偏导,可得
    $$
    \begin{cases}
        u_y' = f_1' \cdot u_y' \cdot x + f_2' \cdot (v_y'+1),\\
        v_y' = g_1' \cdot u_y' + g_2' \cdot (2 v v_y' y + 2v^2).
    \end{cases}
    $$
    标准化为
    $$
    \begin{cases}
        (x f_1' - 1) u_y' + f_2' v_y' = -f_2',\\
        g_1' u_y' + (2v g_2' y - 1) v_y' = 2v g_2'.
    \end{cases}
    $$
    令$D = \begin{vmatrix}
        x f_1' - 1 & f_2'\\
        g_1' & 2v g_2' y - 1
    \end{vmatrix}$,则$D \neq 0$,再令
    $$
    \tilde{D_1} = \begin{vmatrix}
        -f_2' & f_2'\\
        v^2 g_2' & 2v g_2' y - 1
    \end{vmatrix}, \quad \tilde{D_2} = \begin{vmatrix}
        x f_1' - 1 & -f_2'\\
        g_1' & g_1'v^2
    \end{vmatrix},
    $$
    由克莱姆法则可得
    $$
    u_y' = \frac{\tilde{D_1}}{D}, \quad v_y' = \frac{\tilde{D_2}}{D}.
    $$
    从而
    $$
    \frac{\partial(u,v)}{\partial(x,y)}
    = \begin{vmatrix}
        u_x' & u_y'\\
        v_x' & v_y'
    \end{vmatrix} = \frac{\begin{vmatrix}
        D_1 & \tilde{D_1}\\
        D_2 & \tilde{D_2}
    \end{vmatrix}}{D}
    $$
\end{solution}

\begin{example}
    设$\begin{cases}
        u = u(x,y),\\
        v = v(x,y)
    \end{cases}$是由方程组
    $$
    \begin{cases}
        F(x,y,u,v) = 0,\\
        G(x,y,u,v) = 0
    \end{cases}
    $$ 确定的隐函数组,$F,G \in C^1$,且$\frac{
        \partial (F,G)}{\partial (u,v)} \neq 0$,求$\dif u, \dif v$.
\end{example}

\begin{solution}
    解法一:

    $\dif u = \parfrac{u}{x} \dif x + \parfrac{u}{y} \dif y$, $\dif v = \parfrac{v}{x} \dif x + \parfrac{v}{y} \dif y$,对$\begin{cases}
        F(x,y,u(x,y),v(x,y)) = 0,\\
        G(x,y,u(x,y),v(x,y)) = 0
    \end{cases}$两边关于$x$求偏导,可得
    $$
    \begin{cases}
        F_x' \cdot 1 + F_u' \cdot \parfrac{u}{x} + F_v' \cdot \parfrac{v}{x} = 0,\\
        G_x' \cdot 1 + G_u' \cdot \parfrac{u}{x} + G_v' \cdot \parfrac{v}{x} = 0.
    \end{cases}
    $$
    \begin{remark}
        助教注:这里的对$F=0$两侧对$x$求偏导,我们区分两个描述
        \begin{enumerate}
            \item $F(x,y,u,v) = 0$,两侧对$x$求偏导,也就是对第一个分量求偏导,得到$F_x' = 0$.
            \item 我们令$\tilde F(x,y) = F(x,y,u(x,y),v(x,y))$,两侧对$x$求偏导,求的是$\frac{\partial \tilde F}{\partial x}$,得到
            $$\frac{\partial \tilde F}{\partial x} = F_x' + F_u' \cdot \parfrac{u}{x} + F_v' \cdot \parfrac{v}{x} = 0$$
        \end{enumerate}
        后者才是隐函数定理的表述,多元函数的隐函数定理是这样的
        \begin{theorem}
            [多元函数隐函数定理]

            设$F:\R^{n+m} \to \R^m$是一个连续可微函数,设$\bm x \in \R^{n},\bm y \in \R^m$.若方程
            $$\begin{cases}
                F^1(\bm x, \bm y) = 0,\\
                F^2(\bm x, \bm y) = 0,\\
                \cdots\\
                F^m(\bm x, \bm y) = 0
            \end{cases}
            $$
            满足在$(\bm {x_0},\bm {y_0})$处,有$F(\bm {x_0},\bm {y_0}) = 0$,且$\frac{\partial (F^1,F^2,\cdots,F^m)}{\partial (\bm y)} \neq 0$,则存在一个邻域$U$和一个函数$$\bm y = \bm \varphi(\bm x) = (\varphi_1(\bm x), \varphi_2(\bm x), \cdots, \varphi_m(\bm x))
            $$
            使得在$U$中,有$\bm y = \bm \varphi(\bm x)$是$\bm x$的函数,且在$U$中有解集可以写为
            \begin{equation}\label{eq:implicit}
                F^1(\bm x, \bm \varphi(\bm x)) = 0, F^2(\bm x, \bm \varphi(\bm x)) = 0, \cdots, F^m(\bm x, \bm \varphi(\bm x)) = 0
            \end{equation}
            我们希望求隐函数的偏导数,是希望求$\bm \varphi = (\varphi_1, \varphi_2, \cdots, \varphi_m)$的偏导数.
            也就是说
            $$\frac{\partial y_1}{\partial x_1} = \frac{\partial \varphi_1}{\partial x_1} $$
            而$\frac{\partial \varphi_1}{\partial x_1}$由10.1求出.
        \end{theorem}
        因此我们在求$\frac{\partial u}{\partial x}$时,我们是对$F(x,y,u(x,y),v(x,y))$两侧对$x$求偏导.
    \end{remark}
    \begin{remark}
        那有的同学会说,这不对啊,为什么例10.3中就是对$F(x,z)=0$对$x$求偏导时就是把$x,z$视作不相关的自变量呢?
        
        这是因为,如果我们已知$z=z(x)$,然后对$F(x,z(x)) = 0$两侧对$x$求导,得到
        $$F_x' + F_z' \cdot z_x' = 0$$
        还是能得到$z_x' = -\frac{F_x'}{F_z'}$.

        我们继续原来的题目.
    \end{remark}
    标准化为
    $$
    \begin{cases}
        F_u' \parfrac{u}{x} + F_v' \parfrac{v}{x} = -F_x',\\
        G_u' \parfrac{u}{x} + G_v' \parfrac{v}{x} = -G_x'.
    \end{cases}
    $$
    令$D = \begin{vmatrix}
        F_u' & F_v'\\
        G_u' & G_v'
    \end{vmatrix}$,则$D \neq 0$,再令
    $$
    D_1 = \begin{vmatrix}
        -F_x' & F_v'\\
        -G_x' & G_v'
    \end{vmatrix}, \quad D_2 = \begin{vmatrix}
        F_u' & -F_x'\\
        G_u' & -G_x'
    \end{vmatrix},
    $$
    此时注意到$D_1 = \begin{vmatrix}
        F_v' & F_x'\\
        G_v' & G_x'
    \end{vmatrix} = \frac{\partial (F,G)}{\partial (v,x)}$, $D_2 = \begin{vmatrix}
        F_u' & F_v'\\
        G_u' & G_v'
    \end{vmatrix} = \frac{\partial (F,G)}{\partial (u,v)}$,由克莱姆法则可得
    \begin{align*}
        \parfrac{u}{x} &= \frac{D_1}{D} = \frac{\frac{\partial (F,G)}{\partial (v,x)}}{\frac{\partial (F,G)}{\partial (u,v)}},\\
        \parfrac{v}{x} &= \frac{D_2}{D} = \frac{\frac{\partial (F,G)}{\partial (u,x)}}{\frac{\partial (F,G)}{\partial (u,v)}}.
    \end{align*}
    对原方程组两边对$y$求偏导,同样可得
    \begin{align*}
        \parfrac{u}{y} &= \frac{\frac{\partial (F,G)}{\partial (v,y)}}{\frac{\partial (F,G)}{\partial (u,v)}},\\
        \parfrac{v}{y} &= \frac{\frac{\partial (F,G)}{\partial (u,y)}}{\frac{\partial (F,G)}{\partial (u,v)}}.
    \end{align*}
    从而
    \begin{align*}
        \dif u = \parfrac{u}{x} \dif x + \parfrac{u}{y} \dif y = \frac{\frac{\partial (F,G)}{\partial (v,x)}\dif x + \frac{\partial (F,G)}{\partial (v,y)} \dif y}{\frac{\partial (F,G)}{\partial (u,v)}},\\
        \dif v = \parfrac{v}{x} \dif x + \parfrac{v}{y} \dif y = \frac{\frac{\partial (F,G)}{\partial (u,x)}\dif x + \frac{\partial (F,G)}{\partial (u,y)} \dif y}{\frac{\partial (F,G)}{\partial (u,v)}}.
    \end{align*}
\end{solution}

\begin{solution}
    解法二:

    对原方程两边同时取全微分,可得
    $$
    \begin{cases}
        F_x' \dif x + F_y' \dif y + F_u' \dif u + F_v' \dif v = 0,\\
        G_x' \dif x + G_y' \dif y + G_u' \dif u + G_v' \dif v = 0.
    \end{cases}
    $$
    以$\dif u, \dif v$为变量.依cramer法则,解得
    \begin{align*}
        \dif u = \frac{D_1}{D} = \parfrac{u}{x} \dif x + \parfrac{u}{y} \dif y = \frac{\frac{\partial (F,G)}{\partial (v,x)}\dif x + \frac{\partial (F,G)}{\partial (v,y)} \dif y}{\frac{\partial (F,G)}{\partial (u,v)}},\\
        \dif v = \frac{D_2}{D} = \parfrac{v}{x} \dif x + \parfrac{v}{y} \dif y = \frac{\frac{\partial (F,G)}{\partial (u,x)}\dif x + \frac{\partial (F,G)}{\partial (u,y)} \dif y}{\frac{\partial (F,G)}{\partial (u,v)}}.
    \end{align*}

    其中,$$D_1 = \begin{vmatrix}
        -(F_x' \dif x + F_y' \dif y) & F_v' \\
        -(G_x' \dif x + G_y' \dif y) & G_v'
    \end{vmatrix}, \quad D_2 = \begin{vmatrix}
        F_u' & -(F_x' \dif x + F_y' \dif y)\\
        G_u' & -(G_x' \dif x + G_y' \dif y)
    \end{vmatrix},\quad D = \begin{vmatrix}
        F_u' & F_v'\\
        G_u' & G_v'
    \end{vmatrix}= \frac{\partial (F,G)}{\partial (u,v)}.$$

\end{solution}

\begin{homework}
    ex9.2:31;ex9.3:6,7,8,10,11(1),14.
\end{homework}




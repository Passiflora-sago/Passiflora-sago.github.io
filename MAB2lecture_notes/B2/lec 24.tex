\setcounter{chapter}{23} % 设置章节计数器


















\chapter{重积分的计算与证明}

\section{对称性}

三重积分$I = \iiint_\Omega f(x,y,z) \dif x \dif y \dif z$具有奇偶对称性,其中$f \in C^1(\Omega)$, $\Omega$是有界区域,则有:

\begin{enumerate}
    \item 若$f(x,y,z)$是关于$z$的奇函数,即$f(x,y,-z) = -f(x,y,z)$,
    且$\Omega$关于$z=0$的坐标面对称,则$\iiint_\Omega f(x,y,z) \dif x \dif y \dif z = 0$.
    \item 若$f(x,y,z)$是关于$z$的偶函数,即$f(x,y,-z) = f(x,y,z)$,且$\Omega$关于$z=0$的坐标面对称,则$\iiint_\Omega f(x,y,z) \dif x \dif y \dif z = 2 \iiint_{\Omega^+} f(x,y,z) \dif x \dif y \dif z$,其中$\Omega^+$是$\Omega$在$z=0$上方的部分.
    \item 若$f(x,y,z)$是关于$x,y,z$分别都是偶函数,且$\Omega$关于$x=0,y=0,z=0$的坐标面对称,则$\iiint_\Omega f(x,y,z) \dif x \dif y \dif z = 8 \iiint_{\Omega^+} f(x,y,z) \dif x \dif y \dif z$,其中$\Omega^+$是$\Omega$在第一卦限的部分.
\end{enumerate}

\section{例题}

\begin{example}
    再次计算椭球体:$\Omega: \frac{x^2}{a^2} + \frac{y^2}{b^2} + \frac{z^2}{c^2} \les 1$的体积.
\end{example}

\begin{solution}
设$f(x,y,z) = 1$,则
$$f(x,y,z) = f(-x,y,z) = f(x,-y,z) = f(x,y,-z)$$
即$f(x,y,z)=1$是关于$x,y,z$的偶函数,且$\Omega$关于$x=0,y=0,z=0$的坐标面对称,因此有
$$V = \iiint_\Omega 1 \dif x \dif y \dif z = 8 \iiint_{\Omega^+} 1 \dif x \dif y \dif z$$
其中$\Omega^+$是$\Omega$在第一卦限的部分.

作坐标变换$x = a r \sin \theta \cos \varphi, y = b r \sin \theta \sin \varphi, z = c r \cos \theta$,则
$$\Omega^+ = \{ (r,\theta,\varphi) | r \les 1, \theta \in [0,\frac{\pi}{2}], \varphi \in [0,\frac{\pi}{2}] \}$$
且$\dif x \dif y \dif z = abc r^2 \sin \theta \dif r \dif \theta \dif \varphi$,因此有
$$V = 8 \int_0^{\frac{\pi}{2}} \int_0^{\frac{\pi}{2}} \int_0^1 abc r^2 \sin \theta \dif r \dif \theta \dif \varphi = \frac{4abc \pi}{3}$$
\end{solution}

\begin{example}
    计算曲面$\Sigma:\left( x^2 + y^2 \right)^2 + z^4 = y$围成的区域$\Omega$的体积$V(\Omega) = \iiint_\Omega 1 \dif x \dif y \dif z$.
\end{example}

\begin{solution}

\begin{enumerate}
    \item 
从$\left( x^2 + y^2 \right)^2 + z^4 = y \ges 0$可知,区域$\Omega$仅在半空间$y \ges 0$中存在
\item 由于曲面
$$\Sigma: \left( x^2 + y^2 \right)^2 + z^4 = y$$
与曲面
$$\Sigma': \left( (-x)^2 + y^2 \right)^2 + z^4 = y$$
$$\Sigma'': \left( x^2 + y^2 \right)^2 + (-z)^4 = y$$
表示的曲面是同一个,因此围成的区域$\Omega$也是同一个.因此$\Omega$关于$x=0,z=0$的坐标面对称;
\item 由$f(x,y,z) = 1$关于$x,z$都是偶函数.
\end{enumerate}
故有$$V(\Omega) = 4 \iiint_{\Omega^+} 1 \dif x \dif y \dif z$$
其中$\Omega^+$是$\Omega$在第一卦限的部分.

作球坐标变换
$$\begin{cases}
    x = r \sin \theta \cos \varphi\\
    y = r \sin \theta \sin \varphi\\
    z = r \cos \theta
\end{cases} \Rightarrow \begin{cases}
    \left( r^2 \sin^2 \theta \right)^2 + \left( r^2 \cos^2 \theta \right)^2 \les y = r \sin \theta \sin \varphi \Rightarrow 0 \les r \les \left( \frac{\sin \theta \sin \varphi}{\sin^4 \theta + \cos^4 \theta} \right)^\frac{1}{3} \\
    0 \les \theta \les \frac{\pi}{2}, 0 \les \varphi \les \frac{\pi}{2}
\end{cases}$$

因此\begin{align*}
    V(\Omega) &= 4 \int_0^{\frac{\pi}{2}} \dif \varphi \int_0^{\frac{\pi}{2}} \dif \theta \int_0^{\left( \frac{\sin \theta \sin \varphi}{\sin^4 \theta + \cos^4 \theta} \right)^\frac{1}{3}} r^2 \sin \theta \dif r\\
    &= 4 \int_0^{\frac{\pi}{2}} \dif \varphi \int_0^{\frac{\pi}{2}} \frac13 \left( \frac{\sin \theta \sin \varphi}{\sin^4 \theta + \cos^4 \theta} \right) \sin \theta \dif \theta\\
    &=\frac{4}{3} \int_0^{\frac{\pi}{2}}  \sin \varphi \dif \varphi \cdot \int_0^{\frac{\pi}{2}} \frac{\sin^2 \theta}{\sin^4 \theta + \cos^4 \theta} \dif \theta\\
    &= \frac{4}{3} \int_0^{\frac{\pi}{2}}  \frac{\tan^2 \theta \sec^2 \theta}{1 + \tan^4 \theta} \dif \theta\\
    &= \frac{4}{3} \int_0^{+\infty} \frac{u^2}{1 + u^4} \dif u \quad (u = \tan \theta) \\
    &=\frac{4}{3}\left. \frac{1}{\sqrt 2} \arctan \frac{u - \frac1u}{\sqrt 2} \right|_0^{+\infty} = \frac{4}{3} \cdot \frac{\pi}{2\sqrt 2} = \frac{\sqrt 2 \pi}{3}
\end{align*}




\end{solution}

\begin{example}
    设$(a,b,c) \neq \theta = (0,0,0)$为常向量,$f(x)$为连续函数.$\Omega:x^2 + y^2 + z^2 \les R^2$.证明:
    $$\iiint_\Omega f(ax + by + cz) \dif x \dif y \dif z = \pi \int_{-R}^R f(\lambda u )\left( R^2 - u^2 \right) \dif u, \lambda = \sqrt{a^2 + b^2 + c^2} >0$$
\end{example}

\begin{proof}
    设$A = \begin{pmatrix}
        \frac{a}{\lambda} & \frac{b}{\lambda} & \frac{c}{\lambda} \\
        a_{21} & a_{22} & a_{23} \\
        a_{31} & a_{32} & a_{33} 
    \end{pmatrix}$为正交矩阵,
    即满足$A A^T = I$,作正交变换

    $$
\begin{pmatrix}
    u \\
    v \\
    w 
\end{pmatrix} = A \begin{pmatrix}
    x \\
    y \\
    z
\end{pmatrix} = \begin{pmatrix}
    \frac{a}{\lambda}  & \frac{b}{\lambda} & \frac{c}{\lambda} \\
    a_{21} & a_{22} & a_{23} \\
    a_{31} & a_{32} & a_{33}
\end{pmatrix} \begin{pmatrix}
    x \\
    y \\
    z
\end{pmatrix}
$$

则$u=\frac{a}{\lambda} x + \frac{b}{\lambda} y + \frac{c}{\lambda} z$,且由
$$\left| {AA^T} \right| = \left|A \right|^2 = 1 \Rightarrow \left| A \right| = 1$$
知道Jacobian行列式$$\pdv{(u,v,w)}{(x,y,z)} = \pm 1$$

且$u^2 + v^2 + w^2 = x^2 + y^2 + z^2 \les R^2$,从而

\begin{align*}
    &\iiint_{x^2 + y^2 + z^2 \les R^2} f(ax + by + cz) \dif x \dif y \dif z\\
    =& \int_{-R}^R f(\lambda u) \dif u \iint_{v^2 + w^2 \les R^2 - u^2} \dif v \dif w\\
    =& \pi \int_{-R}^R f(\lambda u) \left( R^2 - u^2 \right) \dif u\\
\end{align*}




\end{proof}

\begin{example}
    $(a,b) \neq \theta = (0,0), \lambda = \sqrt{a^2+b^2} >0$,证明:
    $$\iint_{x^2 + y^2 \les 1} f(ax + by + c) \dif x \dif y = 2 \int_{-1}^1 \sqrt{1-t^2} f(\lambda t + c) \dif t$$
\end{example}

\begin{proof}
    设$A = \begin{pmatrix}
        \frac{a}{\lambda} & \frac{b}{\lambda} \\
        a_2 & b_2 
    \end{pmatrix} $为正交矩阵,即$A^T = A^{-1} \Rightarrow | A | = \pm 1$,于是
    $$\dif x \dif y = \left| \pdv{(x,y)}{(u,v)} \right| \dif u \dif v = \left| A^{-1} \right| \dif u \dif v = \dif u \dif v$$

    作正交变换$$
    \begin{pmatrix}
        u \\
        v
    \end{pmatrix} = A \begin{pmatrix}
        x \\
        y
    \end{pmatrix}
    $$
    则$u^2+v^2 = \begin{pmatrix}
        u & v
    \end{pmatrix} \begin{pmatrix}
        u \\
        v
    \end{pmatrix} = \begin{pmatrix}
        x & y
    \end{pmatrix} A A^T \begin{pmatrix}
        x \\
        y
    \end{pmatrix} = x^2 + y^2 \les 1$

    从而
    \begin{align*}
        &\iint_{x^2 + y^2 \les 1} f(ax + by + c) \dif x \dif y \\
        =& \iint_{u^2 + v^2 \les 1} f(\lambda u + c) \dif u \dif v\\
        =& \int_{-1}^1 f(\lambda u + c) \dif u \iint_{v^2 \les 1 - u^2} \dif v\\
        =& 2 \int_{-1}^1 f(\lambda u + c) \sqrt{1 - u^2} \dif u 
    \end{align*}
\end{proof}


\begin{example}
    证明:
    $$
    \iint_{x^2 + y^2 \les 1} \e^{x^2 + y^2} \dif x \dif y < \left( \int_{-\frac{\sqrt \pi}{2}}^{\frac{\sqrt \pi}{2}} \e^{t^2} \dif t \right)^2
    $$
\end{example}

\begin{proof}
    由奇偶对称性可知
    $$\iint_{x^2 + y^2 \les 1} \e^{x^2 + y^2} \dif x \dif y = 4 \iint_{D_{11}} \e^{x^2 + y^2} \dif x \dif y$$
    $$\iint_{D_2} \e^{x^2 + y^2} \dif x \dif y = 4 \iint_{D_{21}} \e^{x^2 + y^2} \dif x \dif y$$
    其中$$D_1 : x^2 + y^2 \les 1, D_2 : |x|,|y| \les \frac{\sqrt \pi}{2}$$
    $D_{11}$为$D_1$在第一象限的部分,$D_{21}$为$D_2$在第一象限的部分.取$D_0 = D_{11} \cap D_{21}$,则$$S(D_1) = \pi = S(D_2) \Rightarrow S(D_{11}) = S(D_{21}) \Rightarrow S(D_{11} - D_0) = S(D_{21} - D_0)$$
    而\begin{align*}
        \iint_{D_{11} - D_0} \e^{x^2 + y^2} \dif x \dif y < \iint_{D_{11} - D_0} \e^1 \dif x \dif y  = \iint_{D_{21} - D_0} \e^1 \dif x \dif y < \iint_{D_{21}- D_0} \e^{x^2 + y^2} \dif x \dif y
    \end{align*}
    因此
    $$\iint_{D_{11}} \e^{x^2 + y^2} \dif x \dif y < \iint_{D_{21}} \e^{x^2 + y^2} \dif x \dif y$$
    由此可知$$\iint_{D_1} \e^{x^2 + y^2} \dif x \dif y < \iint_{D_2} \e^{x^2 + y^2} \dif x \dif y$$
\end{proof}



\begin{homework}
    \begin{enumerate}
        \item 用五种方法计算$\Omega: \frac{x^2}{a^2} + \frac{y^2}{b^2} + \frac{z^2}{c^2} \les 1$的体积$V(\Omega)$.
        \item 计算$I = \iiint_{x^2 + y^2 + z^2 \les 1} \cos(ax + by + c) \dif V$与$I = \iiint_{x^2 + y^2 + z^2 \les 1} (ax+by+cz)^m \dif V$其中$(a,b,c) \neq \theta$为常向量$m \in N^*.$
        \item CH10:5,6,8.
    \end{enumerate}
\end{homework}
































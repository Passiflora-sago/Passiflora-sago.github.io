\setcounter{chapter}{1} % 设置章节计数器
\chapter{}

\section{Mar 3 ex8.3:1,2,3.}

\begin{exercise}{8.3.1}
    指出下列方程中那些是旋转曲面,并说明他们是怎样产生的.
        \begin{enumerate}
            \item $\frac{x^2}{4}+\frac{y^2}{9}+\frac{z^2}{9}=1$;
        \begin{solution}
        $\begin{cases}
            \frac{x^2}{4}+\frac{y^2}{9}=1,\\
            z=0.
        \end{cases}$绕$x$轴旋转.
        \end{solution}

            \item $x^2+y^2+z^2=1$;
        \begin{solution}
        $\begin{cases}
            x^2+y^2=1,\\
            z=0.
        \end{cases}$绕$x$轴旋转(其他类似的任意情况合理即可).
        \end{solution}

            \item $x^2+2y^2+34z^2=1$;
        \begin{solution}
        非旋转曲面,是椭球.
        \end{solution}
        
            \item $x^2-\frac{y^2}{4}+z^2=1$;
        \begin{solution}
        $\begin{cases}
            x^2-\frac{y^2}{4}=1,\\
            z=0.
        \end{cases}$绕$y$轴旋转.
        \end{solution}

            \item $\frac{x^2}{9}+\frac{y^2}{16}-\frac{z^2}{9}=1$;
        \begin{solution}
            非旋转曲面,是双叶双曲面.
        \end{solution}          

            \item $x^2-y^2-z^2=1$;
        \begin{solution}
        $\begin{cases}
            x^2-y^2=1,\\
            z=0.
        \end{cases}$绕$x$轴旋转.
        \end{solution}

            \item $x^2+y^2=4z$;
        \begin{solution}
        $\begin{cases}
            x^2=4z,\\
            y=0.
        \end{cases}$绕$z$轴旋转.
        \end{solution}

            \item $\frac{x^2}{4}+\frac{y^2}{9}=3z$.
        \begin{solution}
        非旋转曲面,是双曲抛物面.
        \end{solution}

        \end{enumerate}
\end{exercise}

\begin{exercise}{8.3.2}
    指出下列方程在平面直角坐标系$Oxy$和空间直角坐标系$Oxyz$中分别表示怎样的几何图形.
    \begin{enumerate}
        \item $x=2$;
        \begin{solution}
            \begin{multicols}{2}
            \begin{enumerate}
                \item $Oxy$平面上的一条直线;
                \item $Oxyz$空间中的一个平面.
            \end{enumerate}
            \end{multicols}
        \end{solution}

        \item $y=x+1$;
        \begin{solution}
            \begin{multicols}{2}
            \begin{enumerate}
                \item $Oxy$平面上的一条直线;
                \item $Oxyz$空间中的一个平面.
            \end{enumerate}
            \end{multicols}
        \end{solution} 

        \item $x^2+y^2=4$;
        \begin{solution}
            \begin{multicols}{2}
            \begin{enumerate}
                \item $Oxy$平面上的一个圆;
                \item $Oxyz$空间中的一个圆柱面.
            \end{enumerate}
            \end{multicols}
        \end{solution}

        \item $x^2-y^2=1$;
        \begin{solution}
            \begin{multicols}{2}
            \begin{enumerate}
                \item $Oxy$平面上的一条双曲线;
                \item $Oxyz$空间中的一个双曲柱面.
            \end{enumerate}
            \end{multicols}
        \end{solution}

        \item $y=x^2+1$;
        \begin{solution}
            \begin{multicols}{2}
            \begin{enumerate}
                \item $Oxy$平面上的一条抛物线;
                \item $Oxyz$空间中的一个抛物柱面.
            \end{enumerate}
            \end{multicols}
        \end{solution}

        \item $\begin{cases}
            5x-y+1=0,\\
            2x-y-3=0.
        \end{cases}$;
        \begin{solution}
            \begin{multicols}{2}
            \begin{enumerate}
                \item $Oxy$平面上的点$(-\frac{4}{3},-\frac{11}{3})$;
                \item $Oxyz$空间中的直线.
            \end{enumerate}
            \end{multicols}
        \end{solution}

        \item $\begin{cases}
            \frac{x^2}{4}+\frac{y^2}{9}=1,\\
            y=2.
        \end{cases}$;
        \begin{solution}
            \begin{multicols}{2}
            \begin{enumerate}
                \item $Oxy$平面上的两点$(\pm\frac{\sqrt{20}}{3},2)$;
                \item $Oxyz$空间中的两条直线.
            \end{enumerate}
            \end{multicols}
        \end{solution}

        \item $\begin{cases}
            \frac{x^2}{4}-\frac{y^2}{9}=1,\\
            x=4.
        \end{cases}$.
        \begin{solution}
            \begin{multicols}{2}
            \begin{enumerate}
                \item $Oxy$平面上的两点$(4,\pm 3\sqrt{3})$;
                \item $Oxyz$空间中的两条直线.
            \end{enumerate}
            \end{multicols}
        \end{solution}
    \end{enumerate}
\end{exercise}

\begin{exercise}{8.3.3}
    求下列旋转曲面的方程,并指出他们的名称.
    \begin{enumerate}
        \item 曲线$\begin{cases}
            y^2-\frac{z^2}{4}=1,\\
            x=0.
        \end{cases}$绕$z$轴旋转一周;
        \begin{solution}
            方程为$x^2+y^2-\frac{z^2}{4}=1$,是单叶双曲面.
        \end{solution}

        \item 曲线$\begin{cases}
            y=\sin x(0\les x\les \pi),\\
            z=0.
        \end{cases}$绕$x$轴旋转一周;
        \begin{solution}
            方程为$y^2+z^2=\sin^2 x$,非二次曲面.
        \end{solution}

        \item 曲线$\begin{cases}
            4x^2+9y^2=36,\\
            z=0.
        \end{cases}$绕$y$轴旋转一周;
        \begin{solution}
            方程为$4x^2+9y^2+4z^2=36$,是椭球.
        \end{solution}
    \end{enumerate}
\end{exercise}

\section{Mar 5 ex8.4:1,2,4(4)(5)(6)(10),8,9,11.}
\begin{exercise}{8.4.1}
    通过坐标系的平移,化简方程$x^2-y^2-z^2-2x+2y+z-1=0$.并指出曲面的类型.
\end{exercise}
\begin{solution}
    令$\begin{cases}
        x'=x-1,\\
        y'=y+1,\\
        z'=z+\frac{1}{2}.
    \end{cases}$则原方程化为$x'^2-y'^2-z'^2=\frac{3}{4}$,是双叶双曲面.
\end{solution}

\begin{exercise}{8.4.2}
    通过坐标变换,化简方程$xy-x+y+z+1=0$.并指出曲面的类型.
\end{exercise}
\begin{solution}
    令$\begin{cases}
        x=\frac{x'-y'}{\sqrt{2}},\\
        y=\frac{x'+y'}{\sqrt{2}},\\
        z=z'.
    \end{cases}$带入原式整理得$x'^2-y'^2-2\sqrt{2}y'+2z'+2=0$.
    再令$\begin{cases}
        x''=x',\\
        y''=y'+\sqrt{2},\\
        z''=z'+2.
    \end{cases}$带入得$x''^2-y''^2=-2z''$,是双曲抛物面.
\end{solution}

\begin{exercise}{8.4.4}
    将下列方程按要求做相应的变换:
    \begin{enumerate}
        \item[(4)]$x^2+y^2-z^2=1$转化为球面坐标系方程;
        \begin{solution}
            令$\begin{cases}
                x=r\sin\theta\cos\varphi,\\
                y=r\sin\theta\sin\varphi,\\
                z=r\cos\theta.
            \end{cases}$带入原式整理得$r^2(\sin^2\theta\cos^2\varphi-\sin^2\theta\sin^2\varphi-\cos^2\theta)=r^2(2\sin^2\theta\cos^2\varphi-1)=1$.
        \end{solution}

        \item[(5)]柱面坐标系方程$r^2+2z^2=4$转化为球面坐标系方程;
        \begin{solution}
            利用柱面坐标系和球面坐标系的关系$r^2=x^2+y^2,z=z$.
            带入原式整理得$x^2+y^2+2z^2=4$.
            然后再令令$\begin{cases}
                x=r\sin\theta\cos\varphi,\\
                y=r\sin\theta\sin\varphi,\\
                z=r\cos\theta.
            \end{cases}$带入得$r^2(\sin^2\theta\cos^2\varphi+\sin^2\theta\sin^2\varphi+2\cos^2\theta)=r^2(1+\cos^2\theta)=4$.
        \end{solution}

        \item[(6)]球面坐标系方程$r=2\cos\varphi$转化为柱面坐标系方程;
        \begin{solution}
            $r(r\sin\theta)=2r\sin\theta\cos\varphi$,即化为平面直角坐标系$\sqrt{x^2+y^2+z^2}\sqrt{x^2+y^2}=2x$,再转化为柱面坐标系$\sqrt{r^2+z^2}r=2r\cos\theta$,化简得到$\sqrt{r^2+z^2}=2\cos\theta$
        \end{solution}
        
        \item[(10)]柱面坐标系方程$r^2\cos 2\theta=z$转化成直角坐标系方程.
        \begin{solution}
            $r^2\cos 2\theta=r^2(\cos^2\theta-\sin^2\theta)=x^2-y^2=z$,即$x^2-y^2=z$.
        \end{solution} 
    \end{enumerate}
\end{exercise}

\begin{exercise}{8.4.8}
    已知$Oxyz$空间中以原点为球心,$a$为半径的球面的参数方程表示为
    $$\begin{cases}
        x=a\sin u\cos v,\\
        y=a\sin u\sin v,\\
        z=a\cos u,\\
    \end{cases}u\in[0,\pi],v\in[0,2\pi)$$
    求$Oxyz$空间中以点$P_0(x_0,y_0,z_0)$为球心,$a$为半径的球面的一个参数方程表示.
\end{exercise}
\begin{solution}
    $$\begin{cases}
        x=x_0+a\sin u\cos v,\\
        y=y_0+a\sin u\sin v,\\
        z=z_0+a\cos u,\\
    \end{cases}u\in[0,\pi],v\in[0,2\pi)$$
\end{solution}

\begin{exercise}{8.4.11}
    求椭球面$$\frac{x^2}{a^2}+\frac{y^2}{b^2}+\frac{z^2}{c^2}=1$$的一个参数方程表示.
\end{exercise}
\begin{solution}
    $$\begin{cases}
        x=a\sin u\cos v,\\
        y=b\sin u\sin v,\\
        z=c\cos u,\\
    \end{cases}u\in[0,\pi],v\in[0,2\pi)$$
\end{solution}

\begin{exercise}{8.4.9}
    分别求单叶双曲面$$\frac{x^2}{a^2}+\frac{y^2}{b^2}-\frac{z^2}{c^2}=1$$和双叶双曲面$$\frac{x^2}{a^2}+\frac{y^2}{b^2}-\frac{z^2}{c^2}=-1$$的一个参数方程表示.
\end{exercise}
\begin{solution}
    给出四种方案
    \begin{enumerate}
        \item 利用双曲函数
        \begin{multicols}{2}
              \begin{enumerate}
            \item[单叶] 
            $\begin{cases}
                x=a\cosh u\cos v,\\
                y=b\cosh u\sin v,\\
                z=c\sinh u,\\
            \end{cases}\\u\in(-\infty,+\infty),v\in[0,2\pi)$
            \item[双叶] 
            $\begin{cases}
                x=a\sinh u\cos v,\\
                y=b\sinh u\sin v,\\
                z=c\cosh u,\\
            \end{cases}\\u\in(-\infty,+\infty),v\in[0,2\pi)$
        \end{enumerate}
        \end{multicols}

        \item 利用$\sec^2x-\tan^2x=1$
        \begin{multicols}{2}
              \begin{enumerate}
            \item[单叶] 
            $\begin{cases}
                x=a\sec u\cos v,\\
                y=b\sec u\sin v,\\
                z=c\tan u,\\
            \end{cases}\\u\in(-\frac{\pi}{2},\frac{\pi}{2}),v\in[0,2\pi)$
            \item[双叶] 
            $\begin{cases}
                x=a\tan u\cos v,\\
                y=b\tan u\sin v,\\
                z=c\sec u,\\
            \end{cases}\\u\in(-\frac{\pi}{2},\frac{\pi}{2}),v\in[0,2\pi)$
        \end{enumerate}
        \end{multicols}

        \item 利用$\csc^2x-\cot^2x=1$
        \begin{multicols}{2}
              \begin{enumerate}
            \item[单叶] 
            $\begin{cases}
                x=a\csc u\cos v,\\
                y=b\csc u\sin v,\\
                z=c\cot u,\\
            \end{cases}\\u\in(0,\pi),v\in[0,2\pi)$
            \item[双叶] 
            $\begin{cases}
                x=a\cot u\cos v,\\
                y=b\cot u\sin v,\\
                z=c\csc u,\\
            \end{cases}\\u\in(0,\pi),v\in[0,2\pi)$
        \end{enumerate}
        \end{multicols}

        \item 利用类似柱坐标系的形式
        \begin{multicols}{2}
              \begin{enumerate}
            \item[单叶] 
            $\begin{cases}
                x=a\sqrt{1+u^2}\cos v,\\
                y=b\sqrt{1+u^2}\sin v,\\
                z=cu,\\
            \end{cases}\\u\in(-\infty,+\infty),v\in[0,2\pi)$
            \item[双叶] 
            $\begin{cases}
                x=a\sqrt{u^2-1}\cos v,\\
                y=b\sqrt{u^2-1}\sin v,\\
                z=cu,\\
            \end{cases}\\|u|\ges 1,v\in[0,2\pi)$
        \end{enumerate}
        \end{multicols}

    \end{enumerate}
\end{solution}

\section{Mar 7 ex9.1:12,13,14(2)(7)(9)(10),15,17(1),18}

\begin{exercise}{9.1.12}
    设$f(x+y,\frac{y}{x})=x^2-y^2(x\neq0)$,求$f(2,3),f(x,y)$.
\end{exercise}
\begin{solution}
    令$$\begin{cases}
        u=x+y,\\
        v=\frac{y}{x},
    \end{cases}x\neq0$$
    注意到$\frac{y}{x}=-1\Leftrightarrow x+y=0$,因此$f(u,v)$的定义域应为$\{(u,v)|u\neq0,v\neq-1\}\cup\{(0,-1)\}$
    
    对于$u\neq0,v\neq-1$时
    可解得$$\begin{cases}
        x=\frac{u}{v+1},\\
        y=\frac{uv}{v+1},
    \end{cases}u\neq0,v\neq-1$$
    因此$f(u,v)=\frac{u^2(1-v^2)}{(v+1)^2}=\frac{u^2(1-v)}{1+v}$.

    而对于$u=0,v=-1$时
    令$x=t\neq0,y=-t$,可得$f(0,-1)=0$
    综上$$f(x,y)=\begin{cases}
        \frac{x^2(1-y)}{1+y},&x\neq0,y\neq-1;\\
        0,&x=0,y=-1;
    \end{cases}$$
    计算$f(2,3)=\frac{2^2(1-3)}{1+2}=-2$.
\end{solution}

\begin{exercise}{9.1.13}
    设$f(x,y)=x^y$,$\varphi(x,y)=x+y$,$\psi(x,y)=x-y$,求$f[\varphi(x,y),\psi(x,y)]$,$\varphi[f(x,y),\psi(x,y)]$,\\$\psi[\varphi(x,y),f(x,y)]$.
\end{exercise}
\begin{solution}\\
    $f[\varphi(x,y),\psi(x,y)]=(x+y)^{x-y}$,\\
    $\varphi[f(x,y),\psi(x,y)]=x^y+x-y$,\\
    $\psi[\varphi(x,y),f(x,y)]=x+y-x^y$.
\end{solution}

\begin{exercise}{9.1.14}
    \begin{enumerate}
        \item[(2)] $\lim_{x\to0,y\to a}\frac{\sin xy}{x}$
        \begin{solution}
            \begin{align*}
                \lim_{x\to0,y\to a}\frac{\sin xy}{x}
                &=\lim_{x\to0,y\to a}\frac{y\sin xy}{xy}\\
                &=\lim_{x\to0,y\to a}y\lim_{x\to0,y\to a}\frac{\sin xy}{xy}\\
                &=a\lim_{u \to 0}\frac{\sin u}{u}\\
                &=a\\
            \end{align*}
        \end{solution}
        \item[(7)] $\lim_{x\to+\infty,y\to+\infty}(x^2+y^2)\e^{-(x+y)}$
        \begin{solution}
            \begin{align*}
                0\les\lim_{x\to+\infty,y\to+\infty}(x^2+y^2)\e^{-(x+y)}
                &\les\lim_{x\to+\infty,y\to+\infty}\frac{6(x^2+y^2)}{(x+y)^3}\\
                &\les\lim_{x\to+\infty,y\to+\infty}\frac{6(x^2+y^2)}{(x^2+y^2)^{\frac{3}{2}}}\\
                &=\lim_{x\to+\infty,y\to+\infty}\frac{6}{\sqrt{x^2+y^2}}\\
                &=\lim_{r\to+\infty}\frac{6}{r}\\
                &=0
            \end{align*}
        \end{solution}
        \item[(9)] $\lim_{x\to0,y\to0}\frac{xy}{\sqrt{xy+1}-1}$
        \begin{solution}
            \begin{align*}
                \lim_{x\to0,y\to0}\frac{xy}{\sqrt{xy+1}-1}
                &=\lim_{u\to0}\frac{u}{\sqrt{u+1}-1}\\
                &=\lim_{u\to0}\frac{u}{\frac{u}{2}}\\
                &=2
            \end{align*}
        \end{solution}
                \item[(10)] $\lim_{x\to0,y\to0}\frac{\sqrt{xy+1}-1}{x+y}$
        \begin{solution}
            \begin{align*}
                \lim_{x\to0,y\to0}\frac{\sqrt{xy+1}-1}{x+y}
                &=\lim_{x\to0,y\to0}\frac{xy}{2(x+y)}\\
                &=\lim_{x\to0,y\to0}\frac{1}{\frac{2(x+y)}{xy}}\\
                &=\lim_{x\to0,y\to0}\frac{1}{\frac{2}{x}+\frac{2}{y}}
                \\
                &=\frac{1}{\lim_{x\to0,y\to0}{\frac{2}{x}+\frac{2}{y}}}
                \\
            \end{align*}
        但$\lim_{x\to0,y\to0}{\frac{2}{x}+\frac{2}{y}}$并不存在,当$y=\frac{x}{kx-1}$时,$\frac{2}{x}+\frac{2}{y}=k$.因此原极限不存在.
        \end{solution}
    \end{enumerate}
\end{exercise}

\begin{exercise}{9.1.15}
    若$x=r\cos\theta,y=r\sin\theta$,问沿怎样的方向$\theta(0\les\theta\les2\pi)$,下列极限存在?
    \begin{enumerate}
        \item $\lim_{r\to0^+}\e^{\frac{1}{{x^2-y^2}}}$,
        \item $\lim_{r\to+\infty}\e^{x^2-y^2}\sin2xy$.
    \end{enumerate}
\end{exercise}
\begin{solution}
    \begin{enumerate}
        \item 
        \begin{align*}
            \lim_{r\to0^+}\e^{\frac{1}{{x^2-y^2}}}
            &=\lim_{r\to0^+}\e^{\frac{1}{{r^2(\cos^2\theta-\sin^2\theta)}}}\\
            &=\lim_{r\to0^+}\e^{\frac{1}{{r^2\cos2\theta}}}\\
            &=\begin{cases}
                +\infty,&\cos2\theta>0\\
                0,&\cos2\theta<0
            \end{cases}
        \end{align*}
        其中,由于$x^2-y^2\neq0$,可得$\cos2\theta\neq0$.\\
        因此当$\theta \in(\frac{\pi}{4},\frac{3\pi}{4})\cup(\frac{5\pi}{4},\frac{7\pi}{4})$时极限存在.
        \item 
        \begin{align*}
            \lim_{r\to+\infty}\e^{x^2-y^2}\sin2xy
            &=\lim_{r\to+\infty}\e^{r^2(\cos^2\theta-\sin^2\theta)}\sin(2r^2\sin\theta\cos\theta)\\
            &=\lim_{r\to+\infty}\e^{r^2\cos2\theta}\sin(r^2\sin2\theta)\\
            &=\begin{cases}
                \text{不存在},&\cos2\theta\ges0\\
                0,&\cos2\theta<0
            \end{cases}
        \end{align*}
        因此当$\theta\in(\frac{\pi}{4},\frac{3\pi}{4})\cup(\frac{5\pi}{4},\frac{7\pi}{4})$时极限存在.
    \end{enumerate}
\end{solution}

\begin{exercise}{9.1.17(1)}
    研究下列函数的连续性$f(x,y)=\begin{cases}
        \frac{xy}{x-y},&x\neq y\\
        0,&x=y
    \end{cases}$
\end{exercise}
\begin{solution}
    \begin{enumerate}
        \item 对于满足$x-y\neq0$的点处显然连续.
        \item 对于$x=y=a\neq0$的点,当$k\neq0$时,则有$\lim_{t\to 0,x=a+t,y=a+kt
        }\frac{xy}{x-y}=\lim_{t\to 0}\frac{(a+t)(a+kt)}{(1-k)t}$不存在.
        \item 对于$(0,0)$点,考虑$\lim_{t\to 0,x=t^2+t,y=t^2-t}\frac{xy}{x-y}=\lim_{t\to 0}\frac{t^4-t^2}{2t^2}=-\frac{1}{2}$与
        $\lim_{t\to 0,x=t,y=t}0=0$可知极限也不存在.
    \end{enumerate}

    综上,在$x\neq y$处连续.
\end{solution}

\begin{exercise}{9.1.18}
    证明:函数$$f(x,y)=\begin{cases}
        \frac{x^2y}{x^4+y^2},&x^2+y^2>0\\
        0,&x^2+y^2=0
    \end{cases}$$
    在点$(0,0)$沿着过此点的每一射线$x=t\cos\alpha,y=t\sin\alpha(0\les t\les+\infty)$连续,即$\lim_{t\to 0}f(t\cos\alpha,t\sin\alpha)=f(0,0)$,但此点在点$(0,0)$处并不连续.
\end{exercise}
\begin{proof}
    $$\lim_{t\to 0}f(t\cos\alpha,t\sin\alpha)=\lim_{t\to 0}\frac{t^3\cos^2\alpha\sin\alpha}{t^4\cos^4\alpha+t^2\sin^2\alpha}=
    \lim_{t\to 0}\frac{t\cos^2\alpha\sin\alpha}{t^2\cos^4\alpha+\sin^2\alpha}=0$$
    $$\lim_{x\to 0,y=kx^2}\frac{x^2y}{x^4+y^2}=\lim_{x\to 0}\frac{kx^4}{x^4+k^2x^4}=\frac{k}{1+k^2}$$
    
    这表明在点$(0,0)$处并不连续.
\end{proof}